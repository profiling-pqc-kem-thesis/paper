\sammanfattning
\noindent
\textbf{Bakgrund.} Människor använder internet för bland annat kommunikation, arbete och bankärenden. Asymmetrisk kryptering möjliggör att denna användning sker säkert genom att erbjuda sekretess och tillit online. Även om dessa algoritmer lär vara säkra från attacker från klassiska datorer, riskerar kvantdatorer att knäcka dessa med Shors algoritm. Därför utvecklas nya kvantsäkra krypton för att mitigera detta problem. \acrfull{nist} har påbörjat en standardiseringsprocess för dessa algoritmer.\newline
\textbf{Syfte.} I detta arbete analyserar vi vilka specialiserade funktioner för kvantsäkra algoritmer som finns i stordator-arkitekturen \gls{ibmz}. Vidare studerar vi prestandan av dessa algoritmer på olika hårdvara för att förstå vilka tekniker som kan öka deras prestanda.\newline
\textbf{Metod.} Vi utför en litteraturstudie för att identifiera vad som är karaktäristiskt för kvantsäkra algoritmers prestanda samt vilka funktioner i \gls{ibmz} som kan möta och accelerera dessa. Vidare applicerar vi en experimentell studie för att analysera den praktiska prestandan av de två framträdande finalisterna \gls{ntru} och \gls{mceliece} på konsument-, moln- och stordatormiljöer.\newline
\textbf{Resultat.} Vi fann att \gls{ibmz} kunde accelerera flera centrala symmetriska primitiver så som \gls{sha3} och \gls{aes} via en hjälpprocessor för kryptografiska funktioner (\acrshort{cpacf}). Även om befintliga hårdvarusäkerhetsmoduler inte stödde några av de undersökta algoritmerna, fann vi att de kan accelerera dem via en på-plats-programmerbar grindmatris (\acrshort{fpga}). Baserat på vår experimentella studie, fann vi att datorer med stöd för avancerade vektoriseringsfunktioner (\gls{avx}) möjlggjorde en signifikant acceleration av kvantsäkra algoritmer. Slutligen identifierade vi att vektoriseringsfunktioner, applikationsspecifika integrerade kretsar (\acrshort{asic}s) och \acrshort{fpga}s är centrala tekniker som kan nyttjas för att accelerera dessa algortmer.\newline
\textbf{Slutsatser.} Gällande beredskapen hos hårdvara för en övergång till kvantsäkra krypton, fann vi att de föreslagna algoritmerna inte presterar närmelsevis lika bra som klassiska algoritmer. Trots att det är sannolikt att de kvantsäkra kryptona fortsatt förbättras innan övergången sker, kan förbättrat hårdvarustöd för snabbare vektoriseringsfunktioner, ökad cachestorlekar och tillägget av polynomoperationer signifikant bidra till att minska påverkan av övergången.

\vspace{1cm}
\noindent
\textbf{Nyckelord:} Asymmetrisk Kryptering, Prestandatest, \gls{x86}, \gls{imbz}, \gls{z15}

\cleardoublepage