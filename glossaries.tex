\setacronymstyle{long-short}
\makeglossaries

%\newglossaryentry{kem} {
%  name=KEM,
%  description={A Key Encapsulation Mechanism}
%}
\newglossaryentry{mceliece}{name={Classic McEliece}, description={A post-quantum Key Encapsulation Mechanism}}
\newglossaryentry{ntru}{name=NTRU, description={A post-quantum Key Encapsulation Mechanism}}
\newglossaryentry{saber}{name=SABER, description={A post-quantum Key Encapsulation Mechanism}}
\newglossaryentry{kyber}{name=Kyber, description={A post-quantum Key Encapsulation Mechanism also referred to as CRYSTALS-KYBER}}
\newglossaryentry{post-quantum}{name={post-quantum}, description={An era after the introduction of useful computers built on quantum mechanics}}
\newglossaryentry{z15}{name={z15}, description={An IBM mainframe}}
\newglossaryentry{ibmz}{name={IBM Z}, description={A processor architecture}}
\newglossaryentry{zos}{name={z/OS}, description={An operating system running on IBM Z}}
\newglossaryentry{x25519}{name={X25519}, description={A key exchange algorithm over the elliptic curve Curve25519}}
\newglossaryentry{curve25519}{name={Curve25519}, description={An elliptic curve}}
\newglossaryentry{rsa}{name=RSA, description={An asymmetric cryptographic algorithm}}
\newglossaryentry{p-256}{name={P-256}, description={A standardized elliptic curve}}
\newglossaryentry{supercop}{name=SUPERCOP, description={A toolkit for measuring the performance of cryptographic software}}
\newglossaryentry{sha}{name=SHA, description={A standardized suite of Permutation-Based Hash and Extendable-Output Functions}}
\newglossaryentry{sha3}{name=SHA-3, description={A standardized suite of Permutation-Based Hash and Extendable-Output Functions}}
\newglossaryentry{shake}{name=SHAKE, description={A standardized, extendable-output hash function of the SHA-3 family}}
\newglossaryentry{keccak}{name=KECCAK, description={The algorithm which was selected for standardization under SHA-3}, see=sha3}
\newglossaryentry{openssl}{name=OpenSSL, description={A freely available cryptographic library}}
\newglossaryentry{des}{name=DES, description={Data Encryption Standard}}
\newglossaryentry{aes}{name=AES, description={Advanced Encryption Standard}}
\newglossaryentry{avx}{name=AVX, description={Advanced Vector Extensions - an instruction set for SIMD on CPUs}, see=simd}
\newglossaryentry{avx2}{name=AVX2, description={An expansion of AVX, also known as AVX-256}, see=avx}
\newglossaryentry{avx512}{name=AVX512, description={An expansion of AVX}, see=avx}
\newglossaryentry{sse}{name=SSE, description={Streaming SIMD Extensions - an instruction set for SIMD on CPUs}, see=simd}
\newglossaryentry{aes-instruction-set}{name={AES (Instruction Set)}, text=AES, description={An instruction set for accelerated AES operations}, see=aes}
\newglossaryentry{aes-ni}{name={AES-NI}, description={An instruction set for accelerated AES operations}, see=aes}
\newglossaryentry{x86}{name=x86, description={A family of processor architectures. Typically used to refer to x86-64 (x64, amd64 etcetera)}}
\newglossaryentry{s390x}{name=s390x, description={A Linux kernel architecture designation for IBM's Z architectures}}
\newglossaryentry{pqcrypto}{name=PQCRYPTO, description={Post-Quantum Cryptography for Long-Term Security}}
\newglossaryentry{ind-cca2}{name={IND-CCA2}, description={Indistinguishability of encryption against adaptively Chosen-Ciphertext Attacks}}
\newglossaryentry{owcpa}{name={OW-CPA}, description={One-Wayness against Chosen-Plaintext Attack}}
\newglossaryentry{qubit}{name={qubit}, description={A Quantum Bit}}
\newglossaryentry{power}{name={POWER}, description={IBM's POWER architecture}}
\newglossaryentry{alice}{name={Alice}, description={A common name for denoting party A in a public-key cryptosystem}}
\newglossaryentry{bob}{name={Bob}, description={A common name for denoting party B in a public-key cryptosystem}}
\newglossaryentry{eve}{name={Eve}, description={A common name for denoting an evil party in a public-key cryptosystem}}

\newacronym{kex}{KEX}{Key Exchange Algorithm}
\newacronym{kem}{KEM}{Key Encapsulation Mechanism}
\newacronym{nist}{NIST}{National Institute of Standards and Technology}
\newacronym{ecdh}{ECDH}{Elliptic-Curve Diffie-Hellman}
\newacronym{ecdhe}{ECDHE}{Ephemeral Elliptic-Curve Diffie-Hellman}
\newacronym{dh}{DH}{Diffie-Hellman}
\newacronym{dhe}{DHE}{Ephemeral Diffie-Hellman}
\newacronym{mdn}{MDN}{Mozilla Developer Network}
\newacronym{ietf}{IETF}{Internet Engineering Task Force}
\newacronym{cpacf}{CPACF}{Central Processor Assist for Cryptographic Function}
\newacronym{nxu}{NXU}{Nest Accelerator Unit}
\newacronym{lpar}{LPAR}{logical partition}
\newacronym{hsm}{HSM}{Hardware Security Module}
\newacronym{hcm}{HCM}{Hardware Cryptographic Module}
\newacronym{fpga}{FPGA}{Field-Programmable Gate Array}
\newacronym{ecc}{ECC}{Elliptic-Curve Cryptography}
\newacronym{jit}{JIT}{Just In Time}
\newacronym{vcpu}{vCPU}{virtual CPU}
\newacronym{tls}{TLS}{Transport Layer Security}
\newacronym{vpn}{VPN}{Virtual Private Network}
\newacronym{vps}{VPS}{Virtual Private Server}
\newacronym{sisd}{SISD}{Single Instruction Single Data}
\newacronym{simd}{SIMD}{Single Instruction Multiple Data}
\newacronym{misd}{MISD}{Multiple Instruction Single Data}
\newacronym{mimd}{MIMD}{Multiple Instruction Multiple Data}
\newacronym{cisc}{CISC}{Complex Instruction Set Computer}
\newacronym{risc}{RISC}{Reduced Instruction Set Computer}
\newacronym{fips}{FIPS}{Federal Information Processing Standard}
\newacronym{pci-payment}{PIC}{Payment Card Industry}
\newacronym{smt}{SMT}{Simultaneous Multithreading}
\newacronym{isa}{ISA}{Instruction Set Architecture}
\newacronym{asic}{ASIC}{Application-Specific Integrated Circuit}
\newacronym{ntt}{NTT}{Number-Theoretic Transform}
\newacronym{fft}{FFT}{Fast Fourier Transform}

% These are all available but not used in references
\newglossaryentry{diffie-hellman}{name={Diffie-Hellman}, description={A key-exchange algorithm}, nonumberlist}
\glsadd{diffie-hellman}