\setacronymstyle{long-short}
\makeglossaries

%\newglossaryentry{kem} {
%  name=KEM,
%  description={A Key Encapsulation Mechanism}
%}
\newglossaryentry{mceliece}{name={Classic McEliece}, description={A post-quantum Key Encapsulation Mechanism}}
\newglossaryentry{ntru}{name=NTRU, description={A post-quantum Key Encapsulation Mechanism}}
\newglossaryentry{post-quantum}{name={post-quantum}, description={An era after the introduction of useful computers built on quantum mechanics}}
\newglossaryentry{z15}{name={z15}, description={An IBM mainframe}}
\newglossaryentry{ibmz}{name={IBM Z}, description={A processor architecture}}
\newglossaryentry{zos}{name={z/OS}, description={An operating system running on IBM Z}}
\newglossaryentry{x25519}{name={X25519}, description={A key exchange algorithm over the elliptic curve Curve25519}}
\newglossaryentry{curve25519}{name={Curve25519}, description={An elliptic curve}}
\newglossaryentry{rsa}{name=RSA, description={An asymmetric crypothraphy algorithm}}
\newglossaryentry{p-256}{name={P-256}, description={A standardized elliptic curve}}
\newglossaryentry{supercop}{name=SUPERCOP, description={A toolkit for measuring the performance of cryptographic software}}
\newglossaryentry{sha3}{name=SHA3, description={A standardized suite of Permutation-Based Hash and Extendable-Output Functions}}
\newglossaryentry{shake}{name=SHAKE, description={A standardized, extendable-output hash function of the SHA3 family}}
\newglossaryentry{keccak}{name=KECCAK, description={The algorithm which won the SHA3 standardization contest}, see=sha3}
\newglossaryentry{openssl}{name=OpenSSL, description={A freely available cryptographic library}}
\newglossaryentry{aes}{name=AES, description={Advanced Encryption Standard}}
\newglossaryentry{crystals-kyber}{name={CRYSTALS-KYBER}, description={A post-quantum Key Encapsulation Mechanism}}
\newglossaryentry{saber}{name={SABER}, description={A post-quantum Key Encapsulation Mechanism}}
\newglossaryentry{avx}{name=AVX, description={Advanced Vector Extensions - an instruction set for SIMD on CPUs}}
\newglossaryentry{avx2}{name=AVX2, description={An expansion of AVX}, see=avx}
\newglossaryentry{avx512}{name=AVX512, description={An expansion of AVX}, see=avx}
\newglossaryentry{aes-instruction-set}{name={AES (Instruction Set)}, text=AES, description={An instruction set for accelerated AES operations}, see=aes}
\newglossaryentry{aes-ni}{name={AES-NI}, description={An instruction set for accelerated AES operations}, see=aes}
\newglossaryentry{x86}{name=x86, description={A family of processor architectures}}
\newglossaryentry{s390x}{name=s390x, description={A Linux kernel architecture designation for IBM's Z architectures}}
\newglossaryentry{pgcrypto}{name=PGCRYPTO, description={Post-Quantum Cryptography for Long-Term Security}}

\newacronym{kex}{KEX}{Key Exchange Algorithm}
\newacronym{kem}{KEM}{Key Encapsulation Mechanism}
\newacronym{nist}{NIST}{National Institute of Standards and Technology}
\newacronym{ecdh}{ECDH}{Elliptic-Curve Diffie-Hellman}
\newacronym{ecdhe}{ECDHE}{Ephemeral Elliptic-Curve Diffie-Hellman}
\newacronym{dh}{DH}{Diffie-Hellman}
\newacronym{dhe}{DHE}{Ephemeral Diffie-Hellman}
\newacronym{mdn}{MDN}{Mozilla Developer Network}
\newacronym{ietf}{IETF}{Internet Engineering Task Force}
\newacronym{cpacf}{CPACF}{Central Processor Assist for Cryptographic Function}
\newacronym{simd}{SIMD}{Single Instruction Multiple Data}
\newacronym{hsm}{HSM}{Hardware Security Module}
\newacronym{hcm}{HCM}{Hardware Cryptographic Module}
\newacronym{fpga}{FPGA}{Field-Programmable Gate Array}
\newacronym{ecc}{ECC}{Elliptic-Curve Cryptography}
\newacronym{jit}{JIT}{Just In Time}
\newacronym{vcpu}{vCPU}{virtual CPU}

% These are all available but not used in references
\newglossaryentry{diffie-hellman}{name={Diffie-Hellman}, description={An asymmetric key-exchange algorithm}, nonumberlist}
\glsadd{diffie-hellman}