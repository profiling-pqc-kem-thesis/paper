\chapter{Preface}
In the final thesis, you need to delete this chapter. Here, we specify some preliminaries that are valid for the whole thesis. Specific tips and guidelines are provided in the following chapters. 

\section{On supervisor feedback}
When you prepare the thesis draft, consider that feedback from supervisors cannot be requested outside regular office hours, \emph{even though submission deadlines might be scheduled on a Sunday}. Hence, avoid requesting feedback on Friday afternoon before the submission deadline or even on the weekend. Supervisors should give feedback in a reasonable time-frame. Planning and adhering to internal draft deadlines help you to receive quality feedback, on time.

\section{On formatting}
Please note that the chapter names and the chapter structure in this template are
just suggestions. There is no ``one-size-fits-all'' structure for all types of theses.
You need to use chapter, section and subsection headers that are adapted to your
particular topic.
%Preferably, you should formulate your headers (and lists in general)
%in so-called parallel (grammatical) form or structure.
%If you do not remember what that means, now is the time to refresh your memory.

Headers as well as regular paragraphs should start at the left margin of a page and
be aligned left and right, as in the paragraphs shown here (i.e. unlike in Word).
There should be no white-space between paragraphs, but the first line of each
paragraph should be indented, except for the first paragraph following a section-,
subsection-, or sub-subsection header.

Please make sure to get your citations and references correct and consistent.
Just copying/pasting information from GoogleScholar or bibliographic databases is insufficient,
since the citation information, in particular from GoogleScholar, is often incorrect and/or incomplete.
Please see the course literature, e.g., \cite{berndtsson2007thesis,evans2014write,glasman2010science,zobel2014writing}
for more information about the handling of citations and references.

\todo[inline]{Notes like this can be useful for your own comments. You can hide all of them
at once by adding \texttt{disable} to the list of parameters to the command
\texttt{\textbackslash usepackage[color=blue!10,textsize=footnotesize,textwidth=25mm]{todonotes}}.}

\section{On thesis structure and length}
A thesis typically follows the structure already provided in this document. However, depending on the content and nature of a thesis, you may find it appropriate to deviate from that structure when reporting results and analysis and separate them into two chapters. In general, the contents of results, analysis and discussion are the following:
\begin{itemize}
    \item Results: objective results (data) without analysis and interpretation
    \item Analysis: objective analysis/interpretation of the results, that is based solely on the collected data
    \item Discussion: interpretation of the results and analysis within the context of the body of knowledge (external to your thesis)
\end{itemize}

It often makes sense to combine results and analysis into one chapter in order to avoid redundancy. However, there are scenarios where it makes sense to separate results from analysis. For example, when you designed a study with two separate research methods and one research question requires you to analyse the results in combination. Then it may make sense to report all results in one chapter and the analysis answering the research questions in another. 

Table~\ref{tab:pl} provides suggestions for a range of page lengths for each thesis chapter. Please note that these are rough estimates for your orientation. The chapters shall however not fall below the minimum length estimates. The complete thesis text, excluding preliminaries, references and appendices, shall not exceed 80 pages. 

\begin{table}[htb]
    \centering
    \begin{tabular}{lc}
        \toprule
        Chapter & Min--Max pages  \\
        \midrule
        \nameref{chp:introduction} & 3--5 \\
        \nameref{chp:relatedwork} & 3--7 \\
        \nameref{chapter:method} & 7--15 \\
        \nameref{chp:results} & 8--20 \\
        \nameref{chp:discussion} & 8--15 \\
        \nameref{chp:conclusions} & 5--10 \\
        \midrule
        TOTAL & 34--72 \\
        \bottomrule
    \end{tabular}
    \caption{Chapter length estimates}
    \label{tab:pl}
\end{table}

In the following, each chapter provides some guidance\footnote{Adapted from
\begin{itemize}[nolistsep]
    \item \url{https://sokogskriv.no/en/writing/structure-and-argumentation/structuring-a-thesis/}
    \item \url{https://thesisguide.org/2014/10/13/thesis-architecture/}
    \item \url{https://guidetogradschoolsurvival.wordpress.com/2011/04/08/how-to-write-related-work/}
    \item \url{https://dissertationgenius.com/12-steps-write-effective-discussion-chapter/}
\end{itemize}} on what is expected as content. Please refer to the evaluation rubrics in the thesis guidelines document \cite{guidelines_DP-BTH} to assess yourself regarding the degree to which your content fulfils the criteria.