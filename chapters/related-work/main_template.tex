\chapter{Related Work}
\label{chp:relatedwork}
\section{On the content}
This chapter collects descriptions of existing works that are related to your work. Related, in this sense, means that it aims to solve the same (or similar) problem or uses the same approach to solve a different problem. This chapter typically reads like a structured list. Each list item summarizes a piece of work (typically a research paper) briefly and \emph{explains the relation to your work}. This last part is absolutely crucial: the reader should not have to figure out the relation him- or herself. Is your work better from some perspective? More generalisable? More performant? Simpler? It is OK if it is not, but you should tell the reader.

A few suggestions to make writing your related work section easier:
\begin{itemize}
    \item Every time you read a paper, write a short summary of the paper and highlight important sections. This way you can read your own recap of the paper to decide if it's applicable instead of relying on the abstract.
    \item Use the reference section of the papers you read to search for other papers to read. If a paper is closely related to your topic, then likely the papers they reference are papers that are also closely related to your topic and you should read them.
    \item When writing a paragraph on a paper, make sure you can answer the question ``how does this relate to my work?'' If you can't, consider not including it. 
    \item Think about the ``bigger picture'' of the works you present/summarize. In which ways are they similar or different? 
\end{itemize}
    
\emph{Tip}: End the chapter with a summary that makes clear how your works fits into the works presented and which gaps it fills.