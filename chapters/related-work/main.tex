\chapter{Related Work}
\label{chapter:related-work}

\todo[inline]{3-5 pages}

\noindent In \cite{viet2020} Dang et. al. the second round of submissions to the \gls{nist} standardization process are evaluated in terms of performance. They analyze six lattice-based \gls{fpga} and \gls{asic} implementations, as well as 12 software-based \gls{kem} implementations on hardware platforms as well as submissions using a software/hardware co-design. They do not focus their performance benchmarks on software implementations. As we solely aim to evaluate the performance of the \gls{post-quantum} \glspl{kem} in software, our aim is different from theirs. Furthermore, we will evaluate the performance of the third round of submissions, which represent almost a year of further progress. Not all algorithms represented in the second round made it through to the third round. Furthermore, several algorithms were changed and some algorithms merged to one. In their work, Dang et. al. did not present data on the software performance of all of the algorithms present in the third round of submissions. For example, the \gls{mceliece} submission is not presented. Furthermore, Dang et. al. only run the implementations on a single \gls{x86} processor which makes the results less generalizable for processors of different generations and architectures. The method used to measure the performance of the software-based algorithms provides a simplified view of the performance as it only takes the elapsed time and cycles into account. We believe a more in-depth method of measurement based in accurate hardware-based counters may provide a more detailed and whole picture.

In \cite{chikouche2018} the aim of the authors was to evaluate the performance of various \gls{post-quantum} public-key schemes for constrained-resources smart mobile devices in terms of computational time, required memory and power consumption. Though a public-key encryption scheme may be converted into a \gls{kem}, their work does not cover such topics. As the main purpose of the \gls{nist} standardization purpose and our work is to study \gls{post-quantum} \glspl{kem}, we believe one needs to focus on the \glspl{kem} themselves, rather than the underlying schemes.

In \cite{kumar2020}, Kumar and Pattnaik discuss the underlying mathematics of the third-round submissions NewHope, Frodo, \gls{ntru}, \gls{kyber}, \gls{saber} and \gls{mceliece}. They further also describe the classical algorithms \gls{dh} and \gls{ecdh}. Their focus on the mathematics and fundamental performance costs provides an up-to-date and low level comparison of the various \glspl{kem}. Given their focus on the algebraic constructs, they do not investigate the practicalities of running the algorithms on real hardware. We believe that it is important to understand the underlying reasons for performance differences found in the submissions, but in order to provide a full understanding on the readiness of today's hardware one should study the practical performance when run in various environments.

In \cite{vambol2017}, Vambol et. al. evaluate two \gls{post-quantum} public-key schemes - McEliece and Niederreiter. As the work was published before the release of the round one \gls{nist} submissions were released, the work does not use the algorithms that are likely to be standardized. At the time of writing it has been almost four years since the work of Vambol et. al. was published. Since then, a lot of progress has been made in the potential future algorithms - as seen in the various rounds of submissions to \gls{nist}. Furthermore, Vambol et. al. focus on the characteristics of the underlying cryptosystem for the proposed \gls{mceliece} \gls{kem}, as such it does not provide a fair view of the performance characteristics of the \gls{kem} itself, which we are interested in. The authors also focus on computational complexity and parameter sizes of the cryptosystems. Though these topics are relevant and interesting, they do not constitute enough evidence of the performance characteristics of the \gls{post-quantum} \glspl{kem} on various hardware, such as the \gls{z15} mainframe.