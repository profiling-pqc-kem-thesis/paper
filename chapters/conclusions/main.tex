\chapter{Conclusions and Future Work}
\label{chapter:conclusion}

\noindent Web traffic, online banking, \glspl{vpn} and messaging applications are secured using public-key cryptography algorithms. The algorithms in use will continue to be secure from attacks from conventional computers for the foreseeable future.  But the rise of quantum computers and algorithms, such as Shor's, threatens the security of the classical algorithms and may completely break the classical algorithms in the near future. These developments have been followed for a long time and progress has been made throughout the years to introduce \gls{post-quantum} algorithms. The \acrfull{nist} published an open call for \gls{post-quantum} \gls{kem} and digital signature algorithm submissions, thus starting their standardization process. Their third round of submissions provided the two prominent \gls{kem} finalists \gls{mceliece} and \gls{ntru}. Although research has been done on the performance of the submissions, we identified a gap in the research. We measured the performance on mainframe hardware. We also measured the performance on consumer and cloud hardware to provide context. With regards to these developments, we have answered the following research questions.

\begin{description}
    \item \textbf{RQ1} What specialized instructions and features applicable for \gls{post-quantum} \acrlong{kem}s are available in \gls{ibmz}?
    
    \item \textbf{RQ2} Does the performance of \gls{post-quantum} \acrlong{kem}s differ between architectures and if so, how?
    
    \item \textbf{RQ3} What techniques may be used to increase the performance of \gls{post-quantum} \acrlong{kem}s?
\end{description}

\noindent\textbf{RQ1}. We focused our work on the \gls{z15}. Though we were unable to target the processor with target-specific optimizations for the \gls{post-quantum} \glspl{kem}, we believe our results provide us with enough information to state that considerable performance gains are to be made on \gls{z15} hardware. As the \gls{z15} features the \gls{cpacf}, several key symmetric primitives such as \gls{sha3} and \gls{aes} may be accelerated. Furthermore, the support for \glspl{hsm} with purpose-built hardware for accelerating these primitives allows for further performance increases. Though the \glspl{hsm} do not support the \gls{post-quantum} \glspl{kem} we studied, they do offer programmability and cryptographic agility, enabling future algorithms to be accelerated once standardized. The \gls{z15} further offers \gls{simd} instructions at sustained 5.2GHz, theoretically allowing for highly performant software implementations of \gls{post-quantum} algorithms. 

\noindent\textbf{RQ2}. By researching the sequential performance of algorithms as well as their throughput, we have collected data on their performance characteristics. Based on our measurements, it was clear that the performance of \gls{post-quantum} \glspl{kem} varied largely between architectures and environments. Modern computers running on \gls{x86} hardware utilizing the \gls{avx2} instruction set, performed considerably better than older \gls{x86} computers without support for \gls{avx}. The performance on the non-vectorized implementations on \gls{z15} was similar to that on \gls{x86} hardware.\todo{Further information here?}

\noindent\textbf{RQ3}. We identified that vectorization of \gls{post-quantum} algorithms makes a key difference in performance based on our measurements. Though not all algorithms saw a significant performance increase when vectorized, hardware implementations in either \glspl{asic} or \glspl{fpga} were found to significantly increase performance, based on our literature study. Our measurements further provide evidence that the algorithmic change of using the semi-systematic form of \gls{mceliece} outperforms the non-systematic variant. To increase the throughput of the \glspl{kem} more threads may be used. We found that \gls{ntru} scaled the best with regards to the number of threads used.

\noindent\textbf{Outlook}. When considering the readiness of hardware for the transition to \gls{post-quantum} \glspl{kem}, we found that the \glspl{kem} submitted to the \gls{nist} in their current state do not perform nearly as well as the classical algorithms. A transition today would therefore result in a noticeable overhead for clients and have an even greater impact on servers. As the algorithms and implementations have improved significantly in a short period of time, it is likely that the software will continue to improve and lower the overhead before the \gls{post-quantum} transition occurs. To further help the transition, processors should see increased performance in \gls{simd} instructions and an increase in cache sizes. Furthermore, processor designers should commit to bring constant-time vector instructions as well as wider and further vector registers to help improve the speed of the \gls{post-quantum} \glspl{kem}. By bringing hardware-accelerated polynomial multiplication, current and future lattice-based cryptosystems may be further optimized. Hardware-based \gls{kem} implementations were found in literature to be practical and performant, making them a prominent solution for lowering the potential performance impact of the transition.

\noindent\textbf{Future work}. Our work measured throughput performance without any regard to how the latency of the algorithms is affected. By studying the latency alongside the throughput, one may gather data to further understand how the two correlate. Furthermore, we decided to limit our work to two \gls{nist} submissions as the time constraint we were under would not enable us to study further algorithms. By applying the method presented in this thesis to further algorithms, we believe one could achieve a broader understanding of the performance of \gls{post-quantum} \glspl{kem} on various hardware. To further increase the amount of data collectable from the \gls{z15} platform, we believe a study of perf-like tools for the platform is warranted. Furthermore, a more complete study targeting \gls{z15}, applying a more direct and practical method with new implementations for the platform, could yield a more representative result for the performance of \gls{z15}.