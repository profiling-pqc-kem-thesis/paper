\chapter{Discussion}
\label{chapter:discussion}

\section{Post-Quantum Cryptography on IBM z15}

\todo[inline]{
-- ECDHE - potentialen vi gärna hade sett generaliseras för mainframe. Argumentera specialserad workload - så som ECDHE

-- Talk about possible future hardware support, the need for cross-platform vectorization etc?

-- Lack of IBM XL (Auto SIMD) makes it hard to use the hardware - must be manually optimized for the target (not like AVX2)
-- Discuss cross-platform SIMD importance, making hardware more developer-friendly

-- CryptoCards?
}

\section{Readiness of Hardware and Adoption of Post-Quantum Key Encapsulation Mechanisms}

Jacobi and Webb \cite{jacobi2020} claim that mainframes process 90\% of the world's credit card transactions. This sensitive information must be kept secure. Therefore you may argue that the transition to post-quantum cryptography needs to happen earlier in the banking sector than in other sectors. How would the transition to post-quantum cryptography affect mainframe users? Our data suggest that the performance hit would be severe, especially on mainframe hardware. Mainframes have an outstanding \gls{ecdh} performance because it takes advantage of \gls{cpacf} sub-processor to accelerate the calculations. But \gls{cpacf} does not support any of the post-quantum cryptographic algorithms. After \gls{nist}'s standardization process is finished, IBM needs to add support for the standardized algorithms to alleviate the performance impact of the transition. 

\todo[inline]{
-- Påverkan i kontext (knyt an introduktion?) Mainframe: IBM pratade om 90\% av airline bookings, hotels etc. every day.

-- Samhällsviktiga tjänster kanske går över till PQC tidigare än konsumentkrypto - måste vara förberedda, påverkar samhällsviktiga tjänster

-- Talk about possible future hardware support, the need for cross-platform vectorization etc?

-- Ersätt ECDHE i TLS etc. med KEM, jämför prestandan. Notera att fler saker kan komma att behöva ändras.
}

\section{The Performance of Post-Quantum Key Encapsulation Mechanisms}

% mceliece - teoretiskt (enligt författarna) sätt är icke-f snabbare än f - men i praktiken är det verkligen inte så. icke-f mycket icke-determinstiskt
According to the \gls{nist} submission of Albrecht et. al. \cite{mceliece2020}, the systematic variant of \gls{mceliece} theoretically performs better during key generation than the non-systematic variant as it requires less computational work per key-generation attempt. They further state that this is true for as long as the systematic variant succeeds in finding a valid key with as few tries as possible. In reality, we found that the systematic variant consistently performs worse than the non-systematic variant. Although theoretically faster, it seems as if the key generation of the systematic variant requires more tries than the authors anticipated. We further found that the systematic variant performs considerably more non-deterministically with a larger standard deviation than that of the systematic variant. Although the implementation provided by in the \gls{nist} submission is not yet standardized, we believe that one should pursuit the semi-systematic variant of \gls{mceliece} if \gls{nist} decides on standardizing it - if solely taking performance into account. Albrecht et. al. further discuss that it is unknown if the performance gain of the semi-systematic variant warrants its more complicated implementation or if users will even benefit from the speedup. As we found that the semi-systematic \gls{mceliece} variant 8192128f performed more than three times as fast as the systematic reference implementation and almost two times as fast as the systematic AVX2 implementation, we believe that the more complex semi-systematic form is indeed warranted when only considering the performance of the \gls{kem}.

% Cloud hardware - dip in performance over time. Counter argument - cloud provider 1 not behaving the same?
When performing our tests on cloud hardware, we anticipated a less consistent result than on dedicated consumer hardware. We believed that, due the virtualized and shared nature of the resources, the cloud environments would yield varied results over time as other users of the system utilized the hardware. We found that the Cloud Provider 2 environment had several performance discrepancies over time when running \glspl{kem} in sequential iterations. We also found, however, that Cloud Provider 1 largely functioned as the dedicated consumer hardware we tested. Although it's difficult to conclude from the small sample of cloud providers in our tests, we argue that there is in fact a non-zero chance that virtualized cloud hardware performs less consistently than dedicated hardware, given that Cloud Provider 2 had performance discrepancies in all of our sequential benchmarks. 

% Modern Laptop - Cache misses, oregelbundna minnesaccess
Another phenomena found in our data is how the Modern Laptop environment consistently yields the largest number of cache misses. Despite having a considerably newer CPU and more available cache than the Old Mid-Range Laptop and the Old Low-Range Laptop, the Modern Laptop environment performed much worse, as seen in Tables \ref{table:results:micro:cache-misses-mceliece-8192128f-enc} and \ref{table:results:micro:cache-misses-ntru-hrss701-enc}. We believe that this is due to the agressive prefetch mechanisms found in newer CPUs \todo{cite}. These mechanism could badly predict what memory is necessary for future computation and as such evict memory that is used by the algorithms we benchmarked. The older machines could have less agressive mechanisms, or lack them all together, leading to fewer faults. We believe that this prefetching of cache did not constitute an issue for the Modern Workstation as it had double the amount of cache, resulting in virtually zero cache misses across the board.

% -- mceliece använder betydligt mycket mer minne än övriga algoritmer - inte lämpligt för IoT etc?
When considering the memory footprint of the proposed \glspl{kem}, one must take into account the stack usage, heap usage as well as the parameter sizes of the algorithms. We found that the heap usage of the algorithms is negligible. The parameter sizes, however, vary considerably between \gls{mceliece} and \gls{ntru}. \gls{mceliece} requires between one and one point three megabytes of memory to store a public key, whilst \gls{ntru} requires a hundredth of that - roughly 1000 bytes. Although \gls{mceliece}'s private key is considerably smaller, it is still ten times as large as \gls{ntru}'s private key at roughly 13KiB. We therefore believe that \gls{mceliece} is impractical for use in low-memory environments such as embedded devices.

% -- ntru skalar mycket bättre än mceliece sett till trådar
The data we collected for throughput and scaling of the algorithms identified that none of the classical algorithms scaled well when increasing the number of threads that concurrently executed the algorithms. In fact, it seemed as if the worst scaling and throughput was found in \gls{dhe}, followed by \gls{mceliece}. The algorithms saw virtually no increase in throughput once the number of threads surpassed the number of cores of the system. The performance of \gls{ecdhe} varied depending on the environment it was in, but in general the scaling was better than that of \gls{dhe}, with improvements made beyond the system's core count. The best scaling by far was found in the AVX2 optimized \gls{ntru} HRSS 701 implementations which saw a near-linear increase in performance with regards to the number of threads - even when passing the core count of the machine itself. Furthermore, the scaling of \gls{ntru} HRSS 701 was found to be the best in Modern Laptop and Modern Workstation, with the latest CPUs included in the test. We therefore strongly believe that \gls{ntru} HRSS 701 is a top candidate for \gls{post-quantum} \glspl{kem}, when only considering performance. Although not strictly comparable, we believe that one may expect similar performance of \gls{ntru} and the classical \gls{ecdhe}, judging from our measurements of throughput and scaling.

\todo[inline]{
-- Talk about how AVX affects the CPU (downclocking) (see background, x86)
}

\section{The Security of Post-Quantum Key Encapsulation Mechanisms}

% -- ntru mycket snabbare än mceliece, men säkerhetskategorin är lägre. HRSS 701 snabbare än HPS4096821, men säkerhetsnivån är lägre.
Our study has disregarded the security of the \gls{nist} submissions, letting us focus on the performance of the algorithms. It goes without saying that the security of the algorithms are of upmost importance. As we have not performed a study of the security of the algorithms on our own, we rely on the information presented by the \gls{nist} submissions themselves. As presented in \ref{table:background:submissions-security-level}, all of the \gls{mceliece} variants we tested are security level 5. The security level of \gls{ntru} varies between 3 and 5 for the HPS 4096821 variant and between 1 and 3 for the HRSS 701 variant (depending on the locality model used). We found that the HRSS 701 variant of \gls{ntru} overall performed the best out of all \gls{kem} algorithms tested. We further found that \gls{mceliece} variants performed the worst. We therefore believe, given our results, that there may be a correlation between the performance and security level of \gls{post-quantum} algorithms.

Although one may believe our sample set is small, we argue that one has to consider the broader picture. As mentioned in \gls{ntru} HPS 4096821 and HRSS 701 originated from two different \gls{nist} submissions. We therefore believe that, in part, these algorithms are different from one another - further increasing the size of the sample set. We do believe, however, that a further study of the correlation is required to definitively state whether or not there is a correlation between the security level of an algorithm and its performance.

% -- Återkoppla till säkerhetskategorierna - kategori 1 är AES128.
% -- Koppla ihop med Grover - AES 256 blir teoretiskt AES 128, vilket gör säkerhetsnivån halveras. I praktiken påstår nist2017 att det inte är ett problem - då den är svår att köra. Teoreitskt kan det då vara ett måste med kategori 5.
As described in section \ref{section:background:security-categories}, the categories are largely based on the strength of classic block-ciphers. Security category 1, for example, is defined based on attacks on AES 128. The fifth and strongest security category, category 5, is defined based on attacks on AES 256. These definitions, as well as the implications of Grover's algorithm described in section \ref{section:background:classical-cryptography-threats}, could result in the categories being redefined if Grover's algorithm becomes practical. Such a redefinition could result in category 5 post-quantum to be as strong as category 1 pre-quantum. That is, one may argue that only category 5 algorithms should be considered true post-quantum algorithms.

\todo[inline]{
-- ntru är otydliga kring säkerhetskategorin, problem med NIST-definitioner? Det krävs ett unisont sätt att mäta allt.
}

\section{On Performance Measurements}

% -- diskutera skillnader i våra mätvärden och NIST submissions. SUPERCOP estimerar, beter sig icke-deterministiskt, vi räknar?.
When studying previous work before outlining the method of this thesis, we identified that SUPERCOP seemed to be the de-facto tool to measure the performance of cryptographic algorithms. It was referenced and used in both the \gls{mceliece} \cite{mceliece2020} and the \gls{ntru} \cite{ntru2020} submissions to \gls{nist}. When studying the source code, we found several areas of concerns which led us to not use the software. One of the reasons was the use of older versions of the \gls{nist} submissions. We were interested in the third round of submissions, whilst SUPERCOP provided the implementations of the second round. Another reason which contributed to our disregard of SUPERCOP was the use of the estimated number of CPU cycles performed as the performance measurement. We identified that the code relied heavily on using the elapsed time and the base frequency of the CPU to calculate the number of required CPU cycles. As we were interested in high-quality measurements of not only CPU-cycles, but also instruction counts, cache misses etcetera, we found the Linux kernel API to be more apt for our use case. As it uses the hardware counters found in many CPUs, we argue that our measurements are accurate since they are counted and not calculated, as is the case in SUPERCOP. Not only did we find the use of perf leading to more deterministic results, but the number of CPU cycles we measured were overall higher than the values found in the \gls{nist} submissions, as presented in Table \ref{table:results:sequential:nist-vs-ours}. We argue that the use of estimated CPU cycles yields inaccurate results which may easily be affected by many variables of the environment, such as the base clock of the CPU. The SUPERCOP user guide recommends that one disable both hyper-threading and boosting so that it may more accurately calculate the number of CPU cycles \cite{supercop}. We believe that it is not a common practice in reality to turn off both hyper-threading and boosting, which led us to use both technologies in our measurements. To decrease the risk of environmental factors such as boosting to affect our measurements, we repeated all of the measurements up to a thousand times per benchmark. We also repeated the entire benchmark at two completely different occasions. We argue that a method of measuring the performance, such as the one used by us in this thesis, would provide the public with more accurate and real-world values regarding the performance of the cryptographic algorithms\todo{Is this a better fit for the validity section?}. Despite us using the Linux kernel perf API, we believe that there are valid reasons use SUPERCOP's method of measuring CPU cycles. The Linux kernel API requires hardware support in order to accurately measure CPU cycles, instruction counts etcetera. This led us to not be able to accurately benchmark all environments with regards to the CPU cycle count. SUPERCOP's solution, however, is less hardware-dependant and as such more suitable for cross-platform measurements on non-Intel hardware and non-Linux operating systems. To help researchers and users achieve more accurate performance measurements, we believe that the industry as a whole could benefit from the addition of hardware-based performance counters in more CPU types and models.

% i3:a, i7:a på ntru-körningar, xeon på mceliece, svårt att jämföra?
Comparing the performance of the \gls{nist} submissions was further complicated by the wide variety of hardware used. The \gls{ntru} submission \cite{ntru2020} used a 3.2 GHz Intel Core i3-6100T for running the reference implementation. The same submission then used an Intel Core i7-4770k to run the AVX2 implementation. The \gls{mceliece} submission \cite{mceliece2020} used an Intel Xeon E3-1275 v3 for its performance measurements. Although we tested a wide range of hardware, we were unable to use the same hardware as the authors. We argue that the fact that a single system was not used for all of the performance benchmarks complicate the comparisons in performance. For example, the Intel Core i3-6100T has 3MB cache \cite{i36100t} and the Xeon E3-1275 v3 8MB \cite{xeon31275}. As we found that the number of cache misses were considerably reduced when the available cache was increased, we believe that it is entirely possible for the performance measurements presented in the \gls{ntru} submission to be considerably worse than if they would have used the same environment as the \gls{mceliece} submission. In part, one may argue that this issue is up to \gls{nist} to solve before considering the standardization of an algorithm, but we believe it makes for a more difficult comparison when such a variety of hardware is used to represent the algorithms' performance no matter the stage of the submissions.

% mceliece Ours: ref-optimized modern workstation, theirs Intel  Xeon ref -O3 etc...
% ntru: ours ref-optimized modern workstation, theirs intel i 3-6000? etc...
% ntru ours: avx2-optimized modern ... theris i7-4770K (Haswell)
% supercop 1% av ntruhps4096821, 60% av ntruhrss701 wtf?
% amd64; CoffeeLake (906ea); 2017 Intel Core i7-8700; 6 x 3200MHz; bitvise, supercop-20190910
% https://bench.cr.yp.to/results-kem.html
% NTRU stated that their results were to be presented on the same page, but they were not

% Threats to Validity
\section{Threats to Validity}

\subsection{Conclusion validity}

\todo[inline]{Add Conclusion validity}

\subsection{Internal validity}
\label{section:method:internal-validity}
% History - An unrelated event influences the outcomes.
Other software running on the OS may influence the results. To prevent this, we minimized the set of programs running on the system and ran the experiment multiple times at different occasions to decrease the risk of indexing tasks etcetera running on the system.

Linux was used for many reasons. One of them is that it does not have any background tasks like Windows update that can significantly impact the results. It is unknown, however, if there are large indexing tasks or the like which may have interfered the results.

% Maturation - The outcomes of the study vary as a natural result of time.
Another factor, \gls{jit} compilation and garbage collection could have influenced the results. As we used the programming languages C, Assembly and Go, this will not be a problem as they do not use \gls{jit} compilation. Other programs running on the system which use \gls{jit} compilation or garbage collection falls under the previous category and will be handled in the same way.

The post-quantum implementations have not been finalized nor standardized, which may have affected the performance of the implementations and the relevance of the results in the future. \gls{dhe} and \gls{ecdhe} have been standardized and optimized for many years and have mature implementations, meaning they are likely fully optimized by this point. We therefore do not believe that our performance measurements reflect the eventual state of the \gls{post-quantum} algorithms, which should be taken into account when comparing values.

% Instrumentation - Different measures are used in the pre-test and post-test phases.
We have written some of the measurement and visualization tools used in this work. The implementations may have shortcomings and OS differences may lead to different results. To ensure the correctness and consistency between platforms we both have reviewed the code and tested it on different platforms to ensure it was working correctly. When using third-party tools, we validated the results using other software to ensure that there were no discrepancies without a natural cause.

% Testing - The pre-test influences the outcomes of the post-test.
Hardware can throttle because of the temperature increase caused by the benchmarks. This could result in worse performance than intended when running consecutive tests. To mitigate this, benchmarks were not be run directly after each other. We decided to wait between tests. We initially hoped to be able to wait for the temperature of the system to normalize to the same level it was before the test was started. As we have performed several hundreds of benchmarks in each environment, with several thousand algorithm invocations per benchmark, we were however unable to wait more than one minute between benchmarks. Given more time, we believe one may have gotten more reliable data if the wait between tests was longer. However, we did perform some initial tests which showed that one minute wait was mostly enough to lower the temperature of the system after running a benchmark. We therefore do not believe that our results have been impacted noticeably by any throttling, unless noted. 

We could not use the same OS for all of our environments, which could cause varying results. This should however not be an issue. We do not aim to pit \gls{x86} against \gls{z15} in terms of hardware or software, rather represent a type of computer and evaluate their readiness. Furthermore, we could not ensure the same version of libraries and compiler such as GCC, Clang and OpenSSL. We provide full transparency in the software versions used and data accumulated to help interested parties further study our results and correlate differences in versions and performance.

% Selection bias - Groups are not comparable at the beginning of the study.
One may argue that our choice of studying the two post-quantum algorithms \gls{ntru} and \gls{mceliece} was biased, but we argue that that point is invalid or at the very least simplified. The algorithms were selected from four finalists in round 3 of \gls{nist}'s standardization process. \gls{ntru}, \gls{kyber}, and \gls{saber} are all lattice-based algorithms, and at most one of these will be standardized \cite{nist2020}. \gls{ntru} was selected based on the feedback \gls{nist} gave each of the participants. \gls{mceliece} was the only non-lattice-based finalist. It also has a long and good reputation \cite{nist2020}. In addition to these four finalists, there are alternate candidates still in round 3. These were not considered. The choice was made to bring down the scope of this work to a manageable level. Given more time, we believe that our method would scale well with an additional set of algorithms. Our tooling and procedures were created with the potential of analyzing further algorithms in mind.

As has been mentioned previously, we rely on \gls{nist} recommendations as they in many cases provide the authoritative recommendation of algorithms used by protocol implementers \cite{nist:round-three-submissions, nist2019}. We also used work from PQCRYPTO \cite{eu2015} to back up some of \gls{nist}'s recommendations. As we have not identified any other standardization process like that of \gls{nist}, we have concluded that relying on their expertise in this context is correct.

For the implementations of \gls{dhe} and \gls{ecdhe}, we only use the underlying algorithms provided by \gls{openssl}. This was done as \gls{openssl} was identified as the main library for these algorithms on the tested platforms. Using other libraries such as BoringSSL might have resulted in different measurements. The implementations available in \gls{openssl} have been rigorously tested and analyzed by the industry over the decades it has seen use. We are therefore confident that, although the exact measurements may differ between libraries, \gls{openssl} provides a solid foundation for our use case as a representation of today's algorithms.

%Attrition - Dropout from participants
% Not applicable? Perhaps all subjects and platforms will not be able to see all optimizations or compiler flags?

\subsection{Construct validity}

% Extent to which the experiment setting actually reflects the construct under study. Treatment reflects the construct of the cause well. Output reflects the construct of the effect well

% Construct validity evaluates whether a measurement tool really represents the thing we are interested in measuring. It’s central to establishing the overall validity of a method.

As previously mentioned in section \ref{section:method:experiment:phase1:variables}, we were interested in measuring throughput-related values such as CPU cycles, instruction count, wall-clock time as well as memory-related measurements such as heap and stack usage. For our measurements, we relied on the standard Linux kernel-based API named perf (perf\_event\_open). The API was introduced in Linux 2.6.31 which was released in 2009 \cite{linux:perf-released}. The API has grown, and as is tradition with the Linux development, each iteration of the API has been reviewed extensively by multiple people throughout the years. We are confident that the API provides as accurate data as the kernel is able to collect. To make the API usable, we used a lightweight instrumentation tool\footnote{\url{https://github.com/profiling-pqc-kem-thesis/perforator}} which allowed us to use the perf API to measure events for specific regions of code. As with other third-party tools, we validated its function by comparing the results to other tools. By using Linux trace APIs to monitor the target binary, we were able to insert measurements around a function call by interrupting the program of the measurement tool. As the target program was frozen during the handling of these measurements, we strongly believe that no overhead added by the measurement tool was included in the end result. By running the instrumented benchmark separately from the benchmarks measuring wall-clock time or memory allocation, we are certain that we achieved accurate values for all of our measurements.

When studying the data amassed after applying our toolset for micro-benchmarks, we found that the value 9223372036854775808 occurred a considerable amount of times. As it was considerably larger than other values and since we were not expecting similar values for completely different events, we analyzed the fault. Given size of the problem space, we were unable to identify the root cause. We found that 0.7\% of the values recorded were affected by this issue and that it likely originates in an incorrect handling of unsigned 64-bit integers as the value is one higher than the maximum number a signed 64-bit integer may store. In order to clarify the error, we marked the data and ignored them in the data presented in this thesis. Given the low number of affected measurements, we feel confident in our handling of these errors. One measurement that did show a considerable amount of errors, however, is those for the region syndrome\_asm. The measurements for the region consisted of 33\% of these erroneous measurements. Other regions consisted of about 2\% errors. All of our data is published alongside this work for further verification efforts from third parties.

% Number of errors:
%     330 gen_e
%     339 poly_R2_inv
%     164 poly_R2_inv_to_Rq_inv
%      49 poly_Rq_inv
%   2293 poly_Rq_mul
%      56 poly_S3_inv
%     713 poly_Sq_mul
%     281 randombytes
%      24 root
%      50 syndrome
%     522 syndrome_asm

% Total amount:
%   16048 gen_e
%   12036 poly_R2_inv
%     4012 poly_R2_inv_to_Rq_inv
%   12036 poly_Rq_inv
%   204504 poly_Rq_mul
%   12036 poly_S3_inv
%   24036 poly_Sq_mul
%   36108 randombytes
%   30673 root
%     8024 syndrome
%     1578 syndrome_asm

\subsection{Content validity}

% Refers to the extent to which a measure represents all facets of a given construct.
%"refers to the degree to which an assessment instrument is relevant to, and representative of, the targeted construct it is designed to measure."

To answer \researchquestion{2}, we did not need to measure any function beyond the main algorithm invocation. This since the algorithm's usage included all of the system's parts. As such, measuring the time, CPU cycles and total instructions of the entire system should have sufficed.

To help answer \researchquestion{1} and \researchquestion{3} we examined optimizations of different parts of the algorithm. We needed to see how the parts in the system as a whole were performing. For this, we used micro-benchmarks. Micro-benchmarks may add a non-trivial overhead that depends on the number of benchmarks. We evaluated the impact of these. We performed the experiment once with and once without the micro-benchmarks to see the difference and make sure it is not statistically significant. As there were not a significant amount of overhead, each micro-benchmark did not need to be run in isolation.

When measuring memory usage, we did not measure any other performance metric as it could have induced a higher memory load unrelated to the algorithm under test. Instead, we performed the measurement of the memory usage of the key-pair generation, encryption, and decryption separate from the CPU measurements.

To measure the wall-clock duration of the algorithms, we used the highest resolution monotonic clock available in each system. We further disabled all other measurements when measuring the time in order to minimize the risk of interference.

\subsection{Criterion validity}

In the publications for the \gls{nist} submissions, the authors have written their own performance analysis using the \gls{supercop} benchmark tool. The presented measurements are for the total number of cycles used by the algorithms for generating keys, encapsulation and decapsulation. As these figures have not been validated by a third party, we did not use them to validate ours. We did however present comparisons in order to identify potential issue in their measurements or ours.

\subsection{External validity}

% Sampling bias.
As previously discussed under section \ref{section:method:internal-validity}, one may argue that the selection of subjects was biased as we do not take the entire population of post-quantum and classical algorithms into account. Such a comparison would however be unfeasible. That is why the presented sample is based on the accumulative recommendations of several organizations. We therefore argue that our sample is representative of the algorithms that are and likely will be in use.

% History.
Another factor that may hurt the generalizability of the results is the potential of a series of unrelated events influencing the outcome. We have identified several actions to help mitigate this risk, as discussed further in section \ref{section:method:internal-validity}.

% Experimenter effect.
The implementations used for benchmarks were not written by us. They were, however, slightly altered in order to support various forms of optimization. This fact may result in the tested implementations performing differently than if the original implementers would have applied the optimizations. It is therefore plausible that the measurements of the samples will not be general to other implementations using the same techniques.

% Aptitude treatment.
As there were several optimization techniques such as vectorization and compiler flags applied simultaneously, there is a potential that techniques canceled each other out or in other ways impacted the performance negatively. We mitigated this risk by evaluating each form of optimization in isolation, before we combined all techniques into a final optimized implementation. Due to time constraints, we were however unable to apply the same method for Clang. Clang was never used to build the non-optimized variants, only those with all the chosen optimization flags set. We do not believe that using Clang without optimization flags would yield any results which would have the potential to invalidate our results. Given more time, we do however believe it would make for an interesting research topic.

% Situation effect.
Factors such as various settings, time of day and location may have limited the generalizability of the presented findings. We have identified several mitigative actions as defined in section \ref{section:method:internal-validity}.

% Counter threats.
To counter threats across the experiment we aimed to improve replication of the results by enabling others to carry out the experiment on their own. This was done by providing detailed methodology, the used tools and any accumulated data. The data, tools and the tested implementations used are available as open source\footnote{\url{https://github.com/profiling-pqc-kem-thesis}}.

\iffalse
\todo[inline]{
Vi mäter bara Intel x86, ingen AMD. Svårt att generalisera då?
}

% == 1 ==
% Medium?
\todo[inline]{
Vi behövde köra om micro en jäkla massa gånger - främst på gammal / "low-end" hårdvara. Low-end-laptop fick inte komplett data trots många omkörningar.
}

% Låg?
\todo[inline]{
only a single run of callgrind on mceliece and ntru's tests. mceliece is non-deterministic and may behave weird in a single run.
}

% Lågprioriterad - poteniell risk att history-valididet blivit sämre? Pauser mellan borde löst det
\todo[inline]{Validity: vänd på körschemat när man kör så att inte NTRU får köras på natten varje gång - utan att sådana saker slås ut.}

% Low?
\todo[inline]{Svårt att mäta på IBM och i molnet?}

% Låg?
\todo[inline]{diskutera hur benchmarks inte säkerställer att rätt svar ges från algoritmerna. Detta utelämnas p.g.a. prestanda / träffsäkerhet i mätningar. Vi löser det genom att ha tester som är samma kod som benchmarken, fast med validering}

% Låg?
\todo[inline]{
-- Dedicated hardware for mainframes may behave differently
}

% Låg? SSD på alla system. Tillräckligt med RAM, det gick aldrig lågt
\todo[inline]{Swap locks the system while writing/reading to from disk}

% Låg? Conclusion validity? Kan vi anta att våra resultat generaliseras om vi bara testat tre algoritmer (HRSS, HPS, Classic McEliece).
\todo[inline]{We have only selected a subset of potential optimizations?}
\fi