\chapter{Introduction}
\label{chapter:introduction}

% Establish the importance of the (general) field.
% Snäll introduktion som ganska snabbt bygger upp till lite mer komplexa ämnen vi berör, presenterar problemet
% Presenterar allmänintresset

% General topic overview - simplified
People worldwide use the Internet every day for a myriad of things. We shop for food, clothes and services, communicate with family and friends and entertain ourselves using streaming services. Businesses and enterprises rely on the Internet not only to serve their ever-growing customer base, but also to transmit confidential information, personal identifiable information, credit card transactions and more. That the internet traffic is kept secure for people and businesses alike is imperative. Public-key cryptography is a fundamental technology in providing this security~\cite{rfc8446}. Used heavily in the \gls{tls} protocol, \glspl{vpn} and other applications, public-key cryptography serves the purpose of ensuring that a party is who they say they are and that encryption keys may be exchanged on insecure channels without ever jeopardizing the confidentiality of the traffic~\cite{rfc8446}. For now, it is believed that the algorithms in use will continue to be secure from attacks by conventional computers for the foreseeable future. The rise of a new type of computer and set of algorithms, the quantum computer and quantum algorithms, have shown that the fundamental security of today's cryptography is threatened and is likely to be made obsolete in the near future~\cite{bernstein2017,ibm:z15:2019,microsoft2020}.

% General, more in-depth, introduce NIST
Today, many protocols use public-key cryptography to exchange a key between two parties. The recommended public-key cryptography suites are based on one of the following problems being hard to solve; integer factorization and elliptic-curve discrete logarithm~\cite{nsa2015, nist2019}. One of the most prominent threats to the encryption algorithms is Shor's algorithm~\cite{shor1997}. The algorithm has been found to be able to solve the previously mentioned problems. However, it is known to be difficult or impractical to run on a traditional computer, but easy to run efficiently on a quantum computer. Today's quantum computers are not powerful enough to execute Shor's algorithm on the large numbers that are used in modern cryptography. As quantum computers get more powerful and available to more people, the threat increases.

The transition to a new set of algorithms that are not built on the same underlying mathematics is becoming increasingly important. These \textit{\gls{post-quantum}} algorithms, have not yet been standardized. The \acrfull{nist} has started their standardization process with an open call for submissions of \gls{post-quantum} \glspl{kem} and digital signature algorithms. The process has, at the time of writing, gone through three rounds of submissions - with algorithmic changes and performance optimizations made each iteration. During this standardization process, the security and performance of the submissions have been researched for various use cases and in various environments. We have identified a lack of research on the performance of these \gls{post-quantum} \glspl{kem} on mainframe hardware.

% Introduce mainframes and connect to introduction of cryptography
A mainframe is a uniquely engineered computer that is designed to handle a large amount of data and bulk transactions~\cite{mainframes}. Their availability, resilience, high throughput and security are core features~\cite{mainframes}. Although they may not be used directly by most people, mainframes such as those running on \gls{ibmz} are used all around the world to process millions of hotel bookings daily as well as 90\% of airline reservations and 90\% of credit card transactions made every day~\cite{jacobi2020}. The fact that personal information, trade secrets and more are kept secure when transferring them for processing and storage is critical. The use of these modern cryptography algorithms makes mainframes susceptible to the issues imposed by the progress of quantum computing and the use of Shor's algorithm. Due to the vast use of mainframes and their reliance on strong public-key cryptography, it is central to our society that the move to \gls{post-quantum} cryptography can be performed without sacrificing the availability, resilience, high throughput or security of mainframes.

% Describe the specific problem you will address
We have investigated if the transition to \gls{post-quantum} cryptography can be made in the near future, in terms of performance of \gls{post-quantum} \glspl{kem} on consumer, cloud and mainframe hardware. With a study of the performance of \gls{post-quantum} algorithms on various architectures, we provide up-to-date data to shed light on the readiness of hardware for the \gls{post-quantum} transition. This data may be used by individuals and businesses alike to understand how the transition may impact them. We also identify what specialized features of a mainframe computer can be utilized to increase the performance of \gls{post-quantum} \glspl{kem}. The following research questions are answered in this thesis.

\begin{description}
    \item \textbf{RQ1} What specialized instructions and features applicable for \gls{post-quantum} \acrlong{kem}s are available in \gls{ibmz}?\label{rq1}
    
    \item \textbf{RQ2} Does the performance of \gls{post-quantum} \acrlong{kem}s differ between architectures and if so, how?\label{rq2}
    
    \item \textbf{RQ3} What techniques may be used to increase the performance of \gls{post-quantum} \acrlong{kem}s?\label{rq3}
\end{description}

% Outline
\clearpage
\noindent The rest of this thesis is structured as follows.

\begin{description}
    \item \textbf{Chapter \ref{chapter:background}} covers general topics in cryptography and computing related to this work. Not all related topics are described, as the reader is assumed to be accustomed to general terms of computing, such as compilers and how computers work in broad terms. The chapter further provides an overview of the mathematics used in classical and \gls{post-quantum} \glspl{kex} and \glspl{kem}. Furthermore, the chapter describes various computer architectures and performance-related topics.

    \item \textbf{Chapter \ref{chapter:related-work}} discusses research that has been conducted prior to this thesis, on topics related to the performance of \gls{post-quantum} \glspl{kem} or analysis of the performance of various computer architectures.
    
    \item \textbf{Chapter \ref{chapter:method}} describes in detail the method used to provide data and information to help answer the stated research questions.
    
    \item \textbf{Chapter \ref{chapter:results}} presents and analyzes the data and information collected as outlined in the method.
    
    \item \textbf{Chapter \ref{chapter:discussion}} discusses the results and the validity of the method. Potential areas of improvement in terms of new features and instructions are discussed, based on the results previously gathered.
    
    \item \textbf{Chapter \ref{chapter:conclusion}} concludes the thesis and discusses future work.
\end{description}