\chapter{Introduction}
\label{chapter:introduction}

\section{Research Questions}
\label{section:introduction:research-questions}

\noindent\textbf{RQ1} Does the performance of \gls{post-quantum} key encapsulation mechanisms differ between architectures and if so, how?\label{rq1}\hfill\par
\noindent\textbf{RQ2} What specialized instructions and features applicable for \gls{post-quantum} key encapsulation mechanisms are available in \gls{z15} and how are they used in context?\label{rq2}\hfill\par
\noindent\textbf{RQ3} What techniques may be used to increase the performance of \gls{post-quantum} key encapsulation mechanisms?\label{rq3}\hfill\par

\todo[inline]{motivate the chosen research questions}
% Outline
\section{Outline}
\label{section:introduction:outline}

\noindent\textbf{Chapter \ref{chapter:background}} covers general topics in cryptography and computing related to this work. Not all related topics are described, as the reader is assumed to be accustomed to general terms of computing, such as compilers and how a computer work in broad terms. The chapter further provides an overview of the mathematics used in classical and \gls{post-quantum} \glspl{kex} and \glspl{kem}. Furthemore, the chapter describes various computer architectures and performance-related topics.

\noindent\textbf{Chapter \ref{chapter:related-work}} discusses research that has been conducted prior to this thesis, on topics related to the performance of \gls{post-quantum} \glspl{kem}, classical \glspl{kex} and analysis of performance provided by various computer architectures.

\noindent\textbf{Chapter \ref{chapter:method}} describes in detail the method used to provide data and information to help answer the stated research questions.

\noindent\textbf{Chapter \ref{chapter:results}} presents and analyzes the data and information collected as outlined in the method.

\noindent\textbf{Chapter \ref{chapter:discussion}} discusses the results and the validity of the method. Potential areas of improvement in terms of new features and instructions are discussed, based on the results previously gathered.

\noindent\textbf{Chapter \ref{chapter:conclusion}} concludes the work and discusses future work.

\noindent\textbf{Chapter \ref{chapter:background}} covers general topics in cryptography and computing related to this work. Not all related topics are described, as the reader is assumed to be accustomed to general terms of computing, such as compilers and how a computer work in broad terms.

\noindent\textbf{Chapter \ref{chapter:related-work}} ...

\noindent\textbf{Chapter \ref{chapter:method}} describes in detail the method used to answer our research questions.

\noindent\textbf{Chapter \ref{chapter:results}} ...

\noindent\textbf{Chapter \ref{chapter:discussion}} discusses the results and the validity of the experiment.

\noindent\textbf{Chapter \ref{chapter:conclusion}} concludes the work.