\chapter{Introduction}
\label{chp:introduction}  % labels are used for cross references
 

\section{On the content}
Your introduction has two main purposes: 1) to give an overview of the main points of your thesis, and 2) to awaken the reader's interest. It is recommended to rewrite the introduction one last time when the writing is done, to ensure that it connects well with your conclusion.

\emph{Tip}: For a nice, stylistic twist you can reuse a theme from the introduction in your conclusion. For example, you might present a particular scenario in one way in your introduction, and then return to it in your conclusion from a different -- richer or contrasting -- perspective.

The introduction should include:
\begin{itemize}
    \item The background for your choice of theme
    \item A problem statement that defines the scope of your thesis
    \item A schematic outline of the remainder of your thesis 
\end{itemize}

\subsection{Background}
The background sets the general tone for your thesis. It should make a good impression and convince the reader why the theme is important and why your approach is appropriate and relevant. Even so, it should be no longer than necessary.

What is considered a relevant background depends on your field and its traditions. Background information might be historical in nature, or it might refer to previous research or practical considerations. You can also focus on a specific text, thinker or problem.

Academic writing often means having a discussion with yourself (or some imagined opponent). To open your discussion, there are several options available. You may, for example:
\begin{itemize}
    \item Refer to a contemporary event
    \item Outline a specific problem, a case study or an example
    \item Review the relevant research/literature to demonstrate the need for this particular type of research 
\end{itemize}

You can also define here the fundamental concepts your thesis builds on. Your thesis implements a new type of parser generator and uses the term non-terminal symbol a lot? Here is where you define what you mean by it. The key to this chapter is to keep it very, very short. Whenever you can, don't reinvent a description for an established concept, but reference a text book or paper instead.
    
\emph{Tip}: Do not spend too much time on your background and opening remarks before you have started with the main text.

\subsection{Defining the scope of your thesis}
One of the first tasks of a researcher is defining the scope of a study, i.e. its area (theme, field) and the amount of information to be included. Narrowing the scope of your thesis can be time-consuming, but the more you limit the scope, the more interesting a thesis becomes and the easier it will be for you to decide what is relevant to include and what can be excluded. This is because a narrower scope lets you clarify the problem and study it at greater depth, whereas very broad research questions only allow a superficial treatment.

Sometimes you can also clarify the scope by providing an overall research question or a hypothesis that will be tested. More specific sub-questions~/~hypotheses should be formulated and motivated in the \nameref{chp:method} chapter.
    
\subsection{Outline}
The outline gives an overview of the main points of your thesis. It clarifies the structure of your thesis and helps you find the correct focus for your work. The outline can also be used in supervision sessions, especially in the beginning. You might find that you need to restructure your thesis. Working on your outline can then be a good way of making sense of the necessary changes. A good outline shows how the different parts relate to each other, and is a useful guide for the reader.

\chapter{Related Work}
\label{chp:relatedwork}
\section{On the content}
This chapter collects descriptions of existing works that are related to your work. Related, in this sense, means that it aims to solve the same (or similar) problem or uses the same approach to solve a different problem. This chapter typically reads like a structured list. Each list item summarizes a piece of work (typically a research paper) briefly and \emph{explains the relation to your work}. This last part is absolutely crucial: the reader should not have to figure out the relation him- or herself. Is your work better from some perspective? More generalisable? More performant? Simpler? It is OK if it is not, but you should tell the reader.

A few suggestions to make writing your related work section easier:
\begin{itemize}
    \item Every time you read a paper, write a short summary of the paper and highlight important sections. This way you can read your own recap of the paper to decide if it's applicable instead of relying on the abstract.
    \item Use the reference section of the papers you read to search for other papers to read. If a paper is closely related to your topic, then likely the papers they reference are papers that are also closely related to your topic and you should read them.
    \item When writing a paragraph on a paper, make sure you can answer the question ``how does this relate to my work?'' If you can't, consider not including it. 
    \item Think about the ``bigger picture'' of the works you present/summarize. In which ways are they similar or different? 
\end{itemize}
    
\emph{Tip}: End the chapter with a summary that makes clear how your works fits into the works presented and which gaps it fills.