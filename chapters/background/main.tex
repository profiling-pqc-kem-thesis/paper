\chapter{Background}
\label{chapter:background}

\section{Public Key Cryptography}

% Describe what public key cryptography is - fundamentals

\section{Post-Quantum}

% Describe what post-quantum means - the issues, define pre-quantum

\subsection{Shor's Algorithm}

\subsection{Grover's Algorithm}

\section{Pre-Quantum Cryptography}
% Establish today's use of cryptography, RSA and ECC

\subsection{Key Exchange}

Today key exchange algorithms such as \gls{x25519} (in some contexts known as \gls{curve25519}\footnote{\href{https://mailarchive.ietf.org/arch/msg/cfrg/-9LEdnzVrE5RORux3Oo\_oDDRksU/}{https://mailarchive.ietf.org/arch/msg/cfrg/-9LEdnzVrE5RORux3Oo\_oDDRksU/}}), \gls{ecdh}, \gls{ecdhe}, \gls{dh}, \gls{dhe} are used to exchange session keys in TLS\footnote{\href{https://tools.ietf.org/html/rfc8446}{https://tools.ietf.org/html/rfc8446}}, SSH\cite{williams2011}, VPNs such as OpenVPN\footnote{\href{https://openvpn.net/community-resources/openvpn-cryptographic-layer/}{https://openvpn.net/community-resources/openvpn-cryptographic-layer/}}, IPSec\footnote{\href{https://tools.ietf.org/html/rfc2409}{https://tools.ietf.org/html/rfc2409}} and Wireguard\footnote{\href{https://www.wireguard.com/protocol/}{https://www.wireguard.com/protocol/}}. Variants of the mentioned key exchange algorithms are also used in messaging applications such as the Signal protocol\cite{gordon2017}. Some of these algorithms and parameter sets are recommended for use today by organizations such as \gls{nist} and the \gls{ietf}, namely \gls{x25519}\footnote{\href{https://tools.ietf.org/html/rfc7748}{https://tools.ietf.org/html/rfc7748}}, \gls{ecdhe}\cite{nist2019} and \gls{dhe}\cite{nist2019}.

A key exchange algorithm is ...

\section{Post-Quantum Cryptography}

Post-quantum cryptography relies on an entirely different set of underlying mathematical problems...

\subsection{Key Encapsulation Mechanism}

A key encapsulation mechanism algorithm is ...

\section{The NIST Post-Quantum Standardization Process}

\section{Lattice-Based Key Encapsulation Mechanisms}

\subsection{NTRU}

\subsection{CRYSTALS-KYBER}

\subsection{SABER}

\section{Code-Based Key Encapsulation Mechanisms}

\subsection{Classic McEliece}

\section{Performance Optimization}

"Theoretical" topics on performance optimization

\subsection{SIMD}

Vectorization ...

\section{CPU Architectures}

Alla datorer är inte likadana. Olika instruktionsset, o.s.v. Det finns massvis med olika.

\section{Consumer and Server Hardware}

x86, ARM, RISC-V

\subsection{AVX}

\section{Mainframe Hardware}

IBM Z, POWER

\subsection{CPACF}

\subsection{HSM}
