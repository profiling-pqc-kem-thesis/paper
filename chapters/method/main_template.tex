\chapter{Method}
\label{chp:method}
\section{On the content}
Here you specify and motivate your research questions (and relate them to the main research question if you have defined one in the introduction). An important property of a research question is that it can be answered. If not, you have probably come up with a theme or a field, not a question. You can find many guidelines online (e.g., this one\footnote{\url{http://www.robertfeldt.net/advice/guide_to_creating_research_questions.pdf}}) on how to formulate good research questions.

Some tips:
\begin{itemize}
    \item Use interrogative words: how, why, which (factors/situations), in which ways, etc.
    \item Some questions are closed and only invoke concrete/limited answers. Others will open up for discussions and different interpretations.
    Asking ``What?'' is a more closed question than asking ``How?'' or ``In what way?'' Asking ``Why'' means you are investigating the causes of a phenomenon. Studying causality is methodologically demanding.
    \item Feel free to pose partially open questions that allow discussions of the overall theme, e.g., ``In what way\dots''; ``How can we understand [a particular phenomenon]?''
    \item Do not use research questions that are answered by just ``yes'' or ``no'', except when you have a specific hypothesis that you are going to test.
    \item Avoid questions stating that you want ``to know'' something. It is very unlikely that you get to know something definitely after your degree project.
\end{itemize}

The method chapter should not iterate the contents of methodology handbooks. You also do not need to describe the differences between quantitative and qualitative methods, or list all different kinds of validity and reliability. Such general descriptions are only meaningful when you need them to motivate the approach you have taken.

What you \emph{must} do is to show how your choice of design and research method is suited to answering your research question(s). A good approach to motivate your choice is to compare the properties, characteristics and features of different research methods, illustrating why a particular method is (not) well suited to answer a particular research question. You should try to identify reasonable (but not all) research methods, that have at least the potential to be considered as alternatives. In addition, you should demonstrate that you have given due consideration to the validity and reliability of your chosen method. By ``showing'' instead of ``telling'', you demonstrate that you have understood the practical meaning of these concepts. This way, the method section is not only able to tie the different parts of your thesis together, it also becomes interesting to read!

\begin{itemize}
    \item Show the reader what you have done in your study, and explain why. How did you collect the data? Which options became available through your chosen approach (and which not)?
    \item What were your working conditions? What considerations did you have to balance?
    \item Tell the reader \emph{what you did to increase the validity} of
    your research. E.g., what can you say about the reliability in data
    collection? How do you know that you have actually investigated what you 
    intended to investigate? What conclusions can be drawn on this basis? 
    Which conclusions are certain and which are more tentative? Can your 
    results be applied in other areas? Can you generalise? If so, why? If 
    not, why not?
    \item You should aim to describe weaknesses as well as strengths. An excellent thesis distinguishes itself by defending -- and at the same time criticising -- the choices made. Being self-critical increases trustworthiness.
\end{itemize}