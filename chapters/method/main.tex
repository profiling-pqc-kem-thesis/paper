\chapter{Method}
\label{chapter:method}

\section{Selected Methods}
\label{section:method:selected-method}

%Argue for one or more methods that are applicable for our questions...
We used two methods to answer our three research questions - a literature study and an experiment.

A literature study was conducted to answer \researchquestion{1} and \researchquestion{3}. We did so as the scope of the question, as well as our limited access to mainframe hardware, made it difficult for us to perform a practical study of the performance potentials of using various hardware features.

To provide information to help us answer \researchquestion{2}, we conducted an experiment in form of a performance test. An experiment was chosen as we could isolate variables and control the environment in which the study was conducted. Furthermore, we were only interested in the raw performance metrics from different computer platforms. Therefore, a survey would not be relevant because we were not interested in user experience, rather the algorithms' performance on various platforms. Neither was a case study relevant because it does not allow for a controlled environment is mostly used in social sciences~\cite{wohlin2012}.

\section{Literature Study}
\label{section:method:literature-study}

As previously mentioned, we aimed to answer \researchquestion{1} and \researchquestion{3} by conducting a literature study. We limited the scope to focus on \gls{z15} and the four finalists in \gls{nist}'s standardization process, round 3.

By studying the algorithms' underlying mathematics, the authors' own optimizations as well as relevant literature on cryptography optimization, we aimed to identify what parts of the algorithms are possible and suitable to optimize.

When potential areas of improvements were identified, literature was studied to find relevant methods available on \gls{z15}, such as specialized instruction sets and other hardware features. IBM's official documentation as well as research conducted by third parties was also studied. That way, we hoped to get a balanced view of the capabilities of the platform.

To identify relevant research papers, we searched for peer-reviewed papers by using databases such as Scopus and Web of Science. By searching for relevant keywords such as \textit{mainframe cryptography}, \textit{z15 cryptography}, \textit{cpacf performance}, \textit{mainframe cpu cache}, \textit{cryptocards}, \textit{z15 simd}, \textit{z15 memory security} and \textit{z15 alu} we selected papers that seemed relevant based on their title and abstracts. We then read the selected papers in their entirety to determine the quality and relevance.

The selected papers, as well as the the papers submitted to \gls{nist}, was included in a start set. The start set was then used for forward snowballing - a useful technique to find more papers that are relevant to provide information to help answer our research question~\cite{wohlin2014}.

\section{Experiment}
\label{section:method:experiment}

\subsection{Goal}
\label{section:method:experiment:goal}

Our goal with the experiment was to analyze \gls{post-quantum} \glspl{kem} for the purpose of evaluating the readiness of today's consumer, mainframe and cloud hardware with respect to their performance from the point of view of businesses and professionals, in the context of replacing the \glspl{kex} used today in \gls{tls}, SSH, \glspl{vpn} and other applications.

The experiment had two phases. In phase one, we aimed to study the effects of changing independent variables possibly related to the performance of the subjects. By studying the performance characteristics of a single run of the algorithms, we also aimed to be able to identify what implementation of each algorithm was the fastest.

In the second phase, we aimed to test the performance characteristics of the algorithms in a parallel fashion. By evaluating the performance of the algorithms when performed in parallel over multiple threads, we aimed to identify how the underlying platform took advantage of the available hardware to run the best optimized implementation as identified in the first phase.

\subsection{Subjects}
\label{section:method:experiment:subjects}

The test subjects of the experiment were two of the \gls{post-quantum} algorithms as submitted to \gls{nist}'s third round of submissions, namely \gls{mceliece} and \gls{ntru}. Three of the four finalists - \gls{ntru}, \gls{kyber} and \gls{saber}, are lattice-based algorithms. At most one of these will be standardized. We selected \gls{ntru} based on the positive comments it received and the negative comments the other contestants received from \gls{nist}. \gls{mceliece} was chosen as it was the only non-lattice-based finalist and it has a long and excellent reputation~\cite{nist2020}. Furthermore, it is not untypical for the future winners to be identified early in the evaluation process~\cite{viet2020}.

The \gls{nist} submissions under test are presented in Table \ref{table:method:experiment:phase1:test-subjects}. All of the selected parameter sets for the subjects have been recommended by their respective authors in the third round of submissions. In the case of \gls{mceliece}, we also relied on previous work from the EU-funded project \gls{pqcrypto}, which identified that not all parameter sets recommended by the authors are appropriate. That is, \gls{pqcrypto} identified that the key size should be a minimum of one megabyte, meaning some candidates were excluded based on their key sizes~\cite{eu2015}.

\begin{table}[H]
    \centering
    \caption{Test subjects}
    \label{table:method:experiment:phase1:test-subjects}
    \begin{tabularx}{\linewidth}{l c X}
        \toprule
        \thead{Name} & \thead{Parameter Set} & \thead{Comment} \\
        \midrule
        \gls{mceliece} & 6960119 & \\
        \gls{mceliece} & 8192128 & \\
        \gls{mceliece} & 6960119f & Differs in the key-generation only\\
        \gls{mceliece} & 8192128f & Differs in the key-generation only\\
        \gls{ntru} & hrss701 & \\
        \gls{ntru} & hps2048677 & \\
        \bottomrule
    \end{tabularx}

\end{table}

\noindent As previously mentioned in section \ref{section:background:classical-cryptography}, there are several recommended classical \glspl{kex}. These recommended algorithms and parameter sets constituted the control subjects for this study and are presented in Table \ref{table:method:experiment:phase1:control-subjects}. Although the control subjects may not be directly comparable to the \glspl{kem}, we hoped to provide context for how these classical algorithms perform.

\begin{table}[H]
    \centering
    \caption{Control subjects}
    \label{table:method:experiment:phase1:control-subjects}
    \begin{tabularx}{\linewidth}{l X}
        \toprule
        \thead{Name} & \thead{Parameter Set} \\
        \midrule
        \acrshort{ecdhe} & \gls{p-256}, \gls{x25519} \\
        \acrshort{dhe} & \gls{rsa} with 2048 bit keys \\
        \bottomrule
    \end{tabularx}
\end{table}

\noindent In the case of the \gls{nist} submissions, in this context, the submitted implementations were not thought to be production-ready due to them not yet being standardized nor battle-tested. Due to this, as well as our interest in exploring optimization techniques that are applicable for post-quantum cryptography, we explored various ways the implementations may be optimized.

As the control subjects were intended to represent the algorithms already in use, they were thought to be a solved problem. That is, in this context, implementation details were of no concern - an implementation on the native platform was assumed to be optimized and production-ready.

\subsection{Target Environments}
\label{section:method:experiment:environments}

To fulfil the goal of the experiment, several hardware configurations were tested. These configurations were divided into three categories, namely consumer, cloud and mainframe hardware depicted in figures \ref{table:method:experiment:phase1:consumer-hardware}, \ref{table:method:experiment:phase1:server-hardware} and \ref{table:method:experiment:phase1:mainframe-hardware}.

The consumer hardware was chosen to represent different tiers of personal computers in use for different purposes. The cloud hardware was chosen to represent various cloud providers' \gls{vps} offerings - a common target for many applications~\cite{eurostat2021}. The offerings were all \glspl{vcpu} which ran through hypervisors in a shared hosting environment. The selected mainframe hardware was intended to represent the latest mainframes available from IBM at the time of writing.

Where possible and applicable, the environments ran on the same version of Ubuntu (20.04) and used the same versions of GCC (9.3.0) and Clang (10.0.0). The exact version of each tool in each environment is available in the data published alongside this work in a GitHub repository\footnote{\href{https://github.com/profiling-pqc-kem-thesis/data}{https://github.com/profiling-pqc-kem-thesis/data}}.

\begin{table}[H]
    \centering
    \small
    \caption{Consumer hardware}
    \label{table:method:experiment:phase1:consumer-hardware}
    \begin{tabularx}{\linewidth}{X c c c}
        \toprule
        \thead{Label} & \thead{CPU} & \thead{RAM} & \thead{OS}\\
        \midrule
        Modern Workstation & Intel i9-9900k & 32GB 3600MHz DDR4 & Ubuntu 20.04\\
        Modern Laptop & Intel i7-8565u & 16 GB 2600MHz DDR4 & Ubuntu 20.04\\
        Old Mid-Range Laptop & Intel i5-3230m & 8GB 1600MHz DDR3 & Ubuntu 20.04\\
        Old Low-Range Laptop & Intel i3-3120m & 8GB 1600MHz DDR3 & Ubuntu 20.04\\
        \bottomrule
    \end{tabularx}
\end{table}


\begin{table}[H]
    \centering
    \small
    \caption{Cloud hardware}
    \label{table:method:experiment:phase1:server-hardware}
    \begin{tabularx}{\linewidth}{X c c c}
        \toprule
        \thead{Label} & \thead{CPU} & \thead{RAM} & \thead{OS}\\
        \midrule
        Cloud Provider 1\footnotemark & Intel Xeon 8168, 4 vCPU & 4GB & Ubuntu 18.04\footnotemark\\
        Cloud Provider 2\footnotemark & Intel Xeon 5118\footnotemark, 2 vCPU & 4GB & Ubuntu 20.04\\
        \bottomrule
    \end{tabularx}
\end{table}
\addtocounter{footnote}{-4}
\addtocounter{footnote}{1}
\footnotetext{An Azure "Fsv2" Virtual Private Server in USA East}
\addtocounter{footnote}{1}
\footnotetext{The provider offered Ubuntu 20.04, but at an additional cost}
\addtocounter{footnote}{1}
\footnotetext{A DigitalOcean "Basic \$20 Droplet" in Amsterdam}
\addtocounter{footnote}{1}
\footnotetext{DigitalOcean did not disclose the exact CPU model used, but by using the available metrics we found one likely model}

\begin{table}[H]
    \centering
    \small
    \caption{Mainframe hardware}
    \label{table:method:experiment:phase1:mainframe-hardware}
    \begin{tabularx}{\linewidth}{X c c c c}
        \toprule
        \thead{Label} & \thead{Model} & \thead{CPU} & \thead{RAM} & \thead{OS}\\
        \midrule
        IBM Community Cloud\footnotemark & IBM T01 & IBM \gls{z15} 2 cores\footnotemark & 4GB & RHEL 8.3\footnotemark\\
        \bottomrule
    \end{tabularx}
\end{table}
\addtocounter{footnote}{-3}
\addtocounter{footnote}{1}
\footnotetext{A "General Purpose VM" accessed via LinuxOne Community Cloud}
\addtocounter{footnote}{1}
\footnotetext{It is unknown if they are dedicated or virtual}
\addtocounter{footnote}{1}
\footnotetext{The provider did not offer an Ubuntu distribution}

\subsection{Phase One - Profiling}
\label{section:method:experiment:phase1}

In the first phase, we aimed to identify how possibly performance-related independent variables alter the performance characteristics of the subjects. The performance of the algorithms in a single sequential run was evaluated.

\subsubsection{Variables}
\label{section:method:experiment:phase1:variables}

The independent variables that were identified to potentially be related to performance are presented in Table \ref{table:method:experiment:phase1:independent-variables}. There are virtually infinite possible flags. The ones presented are the minimal ones used for optimization in this context and as used by previous work (\gls{supercop}, \gls{nist} publications). The implementation optimizations listed are the optimizations available in the studied implementations.

As we were interested in the performance characteristics of consumer, cloud and mainframe hardware, \gls{fpga} implementations have been left out from the table. Furthermore, older vectorization extensions sets such as \gls{sse} were not considered as they have been superseeded by \gls{avx}.

\begin{table}[H]
    \centering
    \caption{Independent variables}
    \label{table:method:experiment:phase1:independent-variables}
    \begin{tabularx}{\textwidth}{l l X}
        \toprule
        \thead{Group} & \thead{Label} & \thead{Potential Values} \\
        \midrule
        Compiler & Compilers & GCC, Clang, Intel C++ Compiler (ICC), IBM XL \\
        Compiler & Flags & -O0, -O1, ..., -O5, -march, -mtune, -fomit-frame-pointer, -fwrapv, -fPIC, -fPIE, -mcpu, -Qunused-arguments \\
        Implementation & Libraries & \gls{keccak}, \gls{aes-instruction-set}, \acrshort{cpacf} \\
        Implementation & Optimizations & \acrshort{avx}, \gls{avx2}, \gls{avx512}, \acrshort{simd} on \gls{ibmz}, \acrshort{hsm} \\
        Environment & OS & OS type and version, scheduling etc. \\
        Environment & Hardware & CPU model, CPU architecture, available cores, available RAM, RAM model and specification \\
        \bottomrule
    \end{tabularx}
\end{table}

\noindent The dependent variables we identified are presented in the following list.

\begin{itemize}
    \item CPU cycles
    \item retired instruction count
    \item wall-clock duration
    \item cache misses
    \item page faults
    \item heap allocations
    \item stack allocations
    \item logical core utilization
\end{itemize}

\subsubsection{Selected Toolset - Processor}
\label{section:method:experiment:phase1:selected-toolset-processor}
The instrumentation relied on the Linux perf (perf\_event\_open) API. The API is part of the Linux kernel and produces highly accurate measurements, regardless of the underlying hardware - given support. The recorded values for CPU cycles, instructions, etcetera may be counted or sampled. The API functions externally, by monitoring the entire lifetime of a process, or internally by instrumenting a program with specifically measured blocks of code.

As such the Linux perf API was identified as the only applicable tool for profiling the CPU usage of an algorithm. To help aid the usage of the API, we used a small tool\footnote{\href{https://github.com/profiling-pqc-kem-thesis/perforator}{https://github.com/profiling-pqc-kem-thesis/perforator}} which enabled us to monitor the algorithms during their lifetime.

Valgrind and its Callgrind module was identified as usable for identifying hot paths of the reference implementations as the implementation can be run in a working environment with its results general to wherever the implementation may be run. The tool ran on a single iteration of the reference implementations' tests to produce metrics of the call stack. The data was then analyzed manually in KCacheGrind and QCacheGrind to produce data for graphs of the hot paths.

\subsubsection{Rejected Toolset - Processor}
\label{section:method:experiment:phase1:rejected-toolset-processor}
Other sampling-based measurement alternatives that require the tested binary to be instrumented, such as prof, gprof and gperftools are unable to accurately measure IO-bound tasks, the retired instructions or time spent inside of the kernel~\cite{gprof}.

Another, non-sampling-based tool that monitor the entire lifetime of a process, Valgrind and its sub-tool Callgrind was found to not support running all of our samples on all targeted environments and was therefore deemed inappropriate for use as the main instrumentation tool.

Other libraries or APIs such as those available in \gls{supercop} and PAPI were identified via testing and a source code analysis to provide inaccurate and non-deterministic results and were therefore not applicable for this use case. In the case of \gls{supercop}, some measurements were identified to rely on extrapolation and estimation of data which led us to reject the tool.

\subsubsection{Selected Toolset - Memory}
\label{section:method:experiment:phase1:selected-toolset-memory}

To identify the heap usage we resorted to heaptrack, a tool which allowed us to intercept each heap allocation and deallocation. The tool was found to be simple to build and run on all of our target environments, without yielding a significant run-time overhead. Furthermore, the accompanying tools let us easily automate the process of identifying heap usage on a per-function basis.

For stack analysis we aimed to identify what amount of memory was required to use a specific algorithm. To identify the stack usage we resorted to statically analyzing the assembly code and symbol table produced by the compilers used. The symbol table includes the size of each (non-inlined) function and the assembly code could be used to identify the stack allocation made by each function, if any. The script we wrote to solve this automatically is produced is published alongside this work.

\subsubsection{Rejected Toolset - Memory}
\label{section:method:experiment:phase1:rejected-toolset-memory}

Intel's vTune is a profiling suite with support for \gls{x86}. It was found to work well for gathering memory-related metrics of the CPU during execution of a program. However, it was unavailable for the \gls{z15}. It was therefore rejected as we could rely on other tools to yield cross-platform metrics. 

As previously mentioned, Valgrind is a non-sampling-based tool which enables it to provide accurate measurements. Although we found it was not applicable for us to use for processor profiling, it was identified as a common and popular tool for analyzing memory. In the end we found that, although Valgrind comes with a stack analyzer, we were more interested in the pre-allocated stack usage than the run-time stack usage. Furthermore, heaptrack ended up providing simpler tools to work with and helped us identify allocations on a per-function basis automatically.

\subsubsection{Measurements}
\label{section:method:experiment:phase1:measurements}

Using the aforementioned toolset, we monitored the subjects and measured the following values. Note that not all platforms supported measuring all values as some require hardware support.

\begin{itemize}
    \item Retired instruction count
    \item Total cycle count
    \item Page fault count
    \item Cache miss count
    \item Task clock
    \item Heap allocation
    \item Stack usage
    \item Wall clock time
\end{itemize}

\noindent The measurements were taken at different points in the algorithms' lifecycles - such as at the start and end of the algorithm's invocation. The same measurements were also repeated for all of the stages of the algorithms as presented in section \ref{table:method:experiment:phase1:independent-variables}. By first analyzing the hot paths of the \gls{nist} submissions, we also measured the parts of the algorithm that constituted the highest cost. As some variables in the environment were identified, such as the various implementations for \gls{shake} and \gls{aes}, the use of these were measured as well. In some cases, the authors had provided optimized implementations. In such cases, these optimized functions were measured as well.

As we identified that some tools may introduce overhead when monitoring a program, we decided to split the tests into four parts. These parts are presented in Table \ref{table:method:experiment:phase1:benchmark-divisions}. By splitting the benchmarks we aimed to capture accurate values for each measurements without affecting others.

\begin{table}[t]
    \centering
    \caption{Benchmark divisions}
    \label{table:method:experiment:phase1:benchmark-divisions}
    \begin{tabularx}{\linewidth}{l>{\RaggedRight}X>{\RaggedRight\arraybackslash}X}
        \toprule
        \thead{Name} & \thead{Focus} & \thead{Measurements}\\
        \midrule
        Micro benchmark & Measurements on a per-function basis & Retired instruction count, total cycle count, page fault count, cache miss count, task clock \\
        Sequential benchmark & Accurate times & Wall clock time \\
        Heap benchmark & Heap allocation on a per-function basis & Heap allocation \\
        Stack benchmark & Stack allocation on a per-function basis & Stack usage \\
        \bottomrule
    \end{tabularx}
\end{table}

The measurements were repeated 1000 times for each stage of the algorithms and each part of the benchmark. The heap benchmark ran 10 iterations as the heap usage should be deterministic. Furthermore, the cost of running further iterations was deemed too large. If the total duration of a benchmark surpassed 15 minutes, the test was set to timeout and any complete results up until that point recorded. The timeout was calculated based on the time we had allocated for running the benchmarks. To not let one iteration affect another, we let the system under test idle between each benchmark during one minute. From some early testing we identified that one minute was sufficient for the systems to cool down to a level similar to that before. One minute was also the largest amount of time we could afford to wait, as waiting any longer would make the benchmarks surpass our allocated time budget.

For the exact instrumentation used in each subject, please refer to the source code published alongside this work\footnote{\href{https://github.com/profiling-pqc-kem-thesis/subjects}{https://github.com/profiling-pqc-kem-thesis/subjects}}.

\subsubsection{Compiler Configurations}
\label{section:method:experiment:phase1:compiler-configurations}

% The compilers were configured using a series of performance-related flags. The flags, presented in Table \ref{table:method:experiment:phase1:compilers}, are non-exhaustive, but have been identified to imply most of the performance-related options. With IBM XL on Power, for example, -O5 implies automatic \gls{simd} via -qsimd=auto and improved data locality via -qhot~\cite{ibmxl2018}. Using GCC, -O3 implies loop unrolling and jam transformations via -floop-unroll-and-jam, strict alignment of functions via -fstrict-aliasing etcetera~\cite{gcc2021}. We opted against using the flag -ffastmath as it was documented as an unsafe option for programs relying on the compliance of IEEE or ISO specifications for math functions~\cite{gcc2021}.

The compilers were configured using a series of performance-related flags. The flags, presented in Table \ref{table:method:experiment:phase1:compilers}, are non-exhaustive, but were identified to imply most of the performance-related options. Using GCC, -O3 implies loop unrolling and jam transformations via -floop-unroll-and-jam, strict alignment of functions via -fstrict-aliasing etcetera~\cite{gcc2021}. We opted against using the flag -ffastmath as it was documented as an unsafe option for programs relying on the compliance of IEEE or ISO specifications for math functions~\cite{gcc2021}.

In order to see the exact flags used to compile each sample, please refer to the source code for the experiment\footnote{\href{https://github.com/profiling-pqc-kem-thesis}{https://github.com/profiling-pqc-kem-thesis}}.

\begin{table}[H]
    \centering
    \caption{Compiler configurations}
    \label{table:method:experiment:phase1:compilers}
    \begin{tabularx}{\linewidth}{l c X}
        \toprule
        \thead{Label} & \thead{Compiler} & \thead{Optimization flags}\\
        \midrule
        GCC Minimal & GCC & \\
        Clang Minimal & Clang & \\
%        IBM XL Minimal & IBM XL & \\
        GCC Optimized & GCC & -march=native -mtune=native -O3\\
        Clang Optimized & Clang & -march=native -mtune=native -O3\\
%        IBM XL Optimized & IBM XL & -march=native -mtune=native -O5\\
        \bottomrule
    \end{tabularx}
\end{table}

\subsubsection{Implementation Configurations}
\label{section:method:experiment:phase1:implementation-configurations}

The implementation details presented in Table \ref{table:method:experiment:phase1:implementation-configurations} were changed in order to evaluate potential performance differences. As the number of variables would become unfeasible if we were to evaluate each combination of the implementations, we opted for running a reference implementation and an optimized implementation.

\begin{table}[H]
    \centering
    \caption{Implementation configurations}
    \label{table:method:experiment:phase1:implementation-configurations}
    \begin{tabularx}{\linewidth}{l X}
        \toprule
        \thead{Label} & \thead{Description}\\
        \midrule
        Reference Implementation & Use the reference implementation as published in the \acrshort{nist} submissions, or the available \gls{openssl} implementation. Use a plain \gls{keccak} implementation. Use an available \gls{aes} implementation.\\
        \cmidrule[0.05em]{2-2}
        Optimized Implementation & Use the authors' provided \gls{avx2} implementation. Use a \gls{avx2} \gls{keccak} implementation.\\
        \bottomrule
    \end{tabularx}
\end{table}

\subsubsection{Configuration Summary}
\label{section:method:experiment:phase1:configuration-summary}

All of the aforementioned compiler configurations were tested in all of the environments previously listed. On all environments, the reference implementations of all the algorithms under test were evaluated. If an environment supported \gls{avx2}, an \gls{avx2} implementation was tested as well. Due to the potential cost in time of running Clang builds for each implementation, we opted to only use Clang on the optimized implementation.

\subsection{Phase Two - Throughput and Scalability}
\label{section:method:experiment:phase2}

In phase two, the performance characteristics of the best-performing algorithms from the first phase was tested in a parallel environment. We aimed to identify how the underlying platform takes advantage of the available hardware to run the subjects in a parallel fashion. The same toolset as previously presented was used in this phase.

\subsubsection{Variables}
\label{section:method:experiment:phase2:variables}

The independent variables that were identified to potentially be related to performance are presented in Table \ref{table:method:experiment:phase2:independent-variables}. As previously mentioned, the algorithms under test were run in parallel. As the implementations were selected based on the result from the first phase, several independent variables identified in the first phase were omitted and not of interest in the second phase.

\begin{table}[H]
    \centering
    \caption{Independent variables}
    \label{table:method:experiment:phase2:independent-variables}
    \begin{tabularx}{\linewidth}{l>{\RaggedRight}l>{\RaggedRight\arraybackslash}X}
        \toprule
        \thead{Group} & \thead{Label} & \thead{Comment} \\
        \midrule
        Parallelism & Thread Count & Number of threads running sequential algorithms at the same time \\
        Environment & Hardware & CPU model, CPU architecture, available cores, available RAM, RAM model and specification \\
        Environment & OS & OS type and version, scheduling etc. \\
        \bottomrule
    \end{tabularx}
\end{table}

\noindent We identified throughput and hardware utilization to be dependent variables of potential interest. The hardware utilization of the system were however not considered for this test, instead we focused on how the throughput differs between architectures and environments.

\subsubsection{Measurements}
\label{section:method:experiment:phase2:measurements}

Using the aforementioned toolset, we instrumented the subjects and measured the wall clock time whilst running the benchmark. To evaluate the parallel performance behavior of the algorithm and environment, a simple tool was written and used. The tool created a number of threads and let each thread perform the algorithm under test for 100 iterations. The total wall-clock time for all threads to complete their iterations was used to calculate the throughput. As with previous tests, we set a timeout for 15 minutes and waited for one minute between benchmarks.

For the exact instrumentation used in each subject, please refer to the source code published alongside this work\footnote{\href{https://github.com/profiling-pqc-kem-thesis/subjects}{https://github.com/profiling-pqc-kem-thesis/subjects}}.

\subsubsection{Configurations}
\label{section:method:experiment:phase2:configurations}

The best-performing implementations for each subject as identified in phase one were tested on a series of parallelism configurations, on each environment specified in section \ref{section:method:experiment:environments}.

The parallelism configurations are synonymous with the number of threads used and depended on the degree of \gls{smt} and available physical cores on the target environment. The number of threads started of at one and doubled, up to the product of the number of physical cores and the \gls{smt} degree, multiplied by four.

For a machine with two physical cores and a \gls{smt} degree of two, this calculation yields the thread counts $1$, $2$, $4$, $8$ and $16$.