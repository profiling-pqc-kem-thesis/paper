\chapter{Method}
\label{chapter:method}

This chapter explains and motivates the chosen methods in greater depth.

Section \ref{section:method:selected-method} discusses the selected research method. The application of the selected method is then discussed and explained in greater depth in section \ref{section:method:literature-study} as well as section \ref{section:method:experiment}.

The research questions are listed below for the reader's convenience.

\begin{itemize}
    \item Does performance of \gls{post-quantum} cryptography algorithms differ between architectures and if so, how?
    \item What specialized instructions and features applicable for \gls{post-quantum} cryptography are available in \gls{z15} and how are they used in context?
    \item What techniques may be used to increase performance of \gls{post-quantum} cryptography algorithms for different architectures?
\end{itemize}

\section{Selected Methods}
\label{section:method:selected-method}

%Argue for one or more methods that are applicable for our questions...
We are using two methods to answer our three research questions - a literature study and an experiment.

A literature study will be conducted to answer \researchquestion{1}. We have chosen a study based in literature as the scope of the question, as well as our limited access to mainframe hardware makes it difficult for us to perform a practical study in the performance potentials of using various hardware features.\todo{revisit}

To provide information to help us answer \researchquestion{2} as well as \researchquestion{3}, we will conduct a performance test. In this case, an experiment is the most appropriate method. We can isolate variables and control the environment in which the study is conducted. Furthermore, we are only interested in the raw performance metrics from different computer platforms. Therefore, a survey would not be relevant because we are not interested in user experience, just the algorithms' raw performance on various platforms. Neither is a case study relevant because it does not test any hypotheses, and it is mostly used in social sciences\cite{wohlin2012}.

\section{Literature Study}
\label{section:method:literature-study}

As previously mentioned, we aim to answer \researchquestion{2} by conducting a literature study. We will limit the scope to the four finalists in \gls{nist}'s standardization process, round 3.

By studying the algorithms' underlying mathematics, the authors' own optimizations on \gls{x86}, and relevant literature on cryptography optimization, we aim to identify what parts of the algorithms are possible and suitable to optimize.

When potential areas of improvements have been identified, literature will be studied to find relevant methods available on \gls{z15}, such as specialized instruction sets and other hardware features. IBM's official documentation as well as research conducted by third-parties will be studied. That way, we hope to get a balanced view of the capabilities of the platform.

To identify relevant research papers, we will search for peer-reviewed papers in reputable journals, such as IEEE and Springer. By searching for keywords related to our research topic, we will select papers that seem relevant based on their title and abstracts. We will then read the selected papers in their entirety to determine the quality and relevance.
\todo[inline]{Här bör ni istället beskrive sökmetoden, vilka databaser, t ex scopus och web of science, och vilka nyckelord ni kommer att använda.}

The selected papers, as well as the the papers submitted to \gls{nist}, will be included in a start set. The start set will then be used for forward and backward snowballing - a useful technique to find more papers that are relevant to provide information to help answer our research question\cite{wohlin2014}.
\todo[inline]{Kommer ni att ha specificerade inclusion/exclusion criteria?}

\todo[inline]{
Vill svara på What specialized instructions and features applicable for post-quantum cryp-tography are available in z/15 and how are they used in context?

1. Sätt scope för "vad är post-quantum cryptography?" NIST-finalister?
2. Analysera litteratur för vad som är relevant att förbättra med dessa algoritmer (Applicable for post-quantum cryptography? - Studera implementationer i NIST-submissions? Studera rapporten / matten i NIST-submissions?, Lattice-based, (goppa) code-based, i sin tur SHA3 - sponge, bättre uniform random (AES)?) - Sätter scope för applicable for post-quantum
3. Specialized instructions in z/15?
4. Specialized features in z/15?
5. How are they used in context? - Hur definierar vi det här? Bara teoretiskt "de används för matrismultiplikation", eller faktiska implementationer?

"Identifiera att en algoritm är lattice-based, identifiera att lattice bygger på matriser, visa att matriser går att vektorisera enkelt med studier, identifiera att vektorisering är viktigt - z/15 har hårdvara för z/15"

I discussion argumentera saker som borde finnas i framtiden.
}

\section{Experiment}
\label{section:method:experiment}

This section describes and discusses the experiment design and rationale. The experiment has two phases. In phase one, we profile a run of the key exchange, and in phase two, we test how the performance scalability for millions of key exchanges over multiple cores.

\todo[inline]{Describe that there will be two different "experiments", one for single iteration and one for multi-core}

\subsection{Goal}
\label{section:method:experiment-goal}

Our goal with the experiment is to analyze \textbf{\gls{post-quantum} \glspl{kem}} for the purpose of evaluating \textbf{the readiness of today's consumer, server and mainframe hardware} with the respect to their \textbf{performance} from the point of view of \textbf{businesses and professionals}, in the context of \textbf{replacing the \glspl{kex} used today in TLS, SSH, VPNs and other applications.}

% The context is an analysis of a model of a crypto system using a "physical" implementation.

\subsection{Subjects}

The test subjects of the experiment are two of the post-quantum algorithms as submitted to \gls{nist}'s third round of submissions, namely \gls{mceliece} and \gls{ntru}. Three of the four finalists - \gls{ntru}, \gls{crystals-kyber}, and \gls{saber}, are lattice-based algorithms, and at most one of these will be standardized. We selected \gls{ntru} based on the positive comments it received and the negative comments the other contestants received from \gls{nist}. \gls{mceliece} was chosen as it was the only non-lattice-based finalist and has a long and excellent reputation\cite{nist2020}.
\todo[inline]{Finns fler NIST submissions? Varför är dessa bortvalda?\\ har jag inte readn svart på det ovan?????}
The \gls{nist} submissions under test are presented in table \ref{table:method:experiment:test-subjects}. All of the selected parameter sets for the subjects have been recommended by their respective authors in the third round of submissions. In the case of \gls{mceliece}, we also rely on previous work from the EU-funded project \gls{pqcrypto}, which identified that not all parameter sets recommended by the authors are appropriate. That is, \gls{pqcrypto} identified that the key size should be a minimum of one megabyte, meaning some candidates were excluded\cite{eu2015}.

\begin{table}[H]
    \centering
    \caption{Test Subjects}
    \label{table:method:experiment:test-subjects}
    \begin{tabularx}{\linewidth}{l c X}
        \toprule
        \thead{Name} & \thead{Parameter Set} & \thead{Comment} \\
        \midrule
        \gls{mceliece} & 6960119 & \\
        \gls{mceliece} & 8192128 & \\
        \gls{mceliece} & 6960119f & Differs in the key-generation only\\
        \gls{mceliece} & 8192128f & Differs in the key-generation only\\
        \gls{ntru} & hrss701 & \\
        \gls{ntru} & hps2048677 & \\
        \bottomrule
    \end{tabularx}

\end{table}

As previously mentioned in section \ref{section:background:pre-quantum}, there are several recommended classical \glspl{kex}. These recommended algorithms and parameter sets will constitute the control subjects for this study and are presented in table \ref{table:method:experiment:control-subjects}.

\begin{table}[H]
    \centering
    \caption{Control Subjects}
    \label{table:method:experiment:control-subjects}
    \begin{tabularx}{\linewidth}{l X}
        \toprule
        \thead{Name} & \thead{Parameter Set} \\
        \midrule
        \gls{ecdhe} & \gls{p-256}, \gls{x25519} \\
        \gls{dhe} & \gls{rsa} with 2048 bit keys \\
        \bottomrule
    \end{tabularx}
\end{table}

In the case of the \gls{nist} submissions, in this context, the submitted implementations are not thought to be production-ready due to them not yet being standardized nor battle-tested. This, in addition to our interest in exploring optimization techniques that are applicable for post-quantum cryptography, results in us exploring various ways the implementations may be optimized.

As the control subjects are intended to represent the algorithms already in use, they are thought to be a solved problem. That is, in this context, implementation details are of no concern - an implementation on the native platform is assumed to be optimized and production-ready.

\subsection{Variables}

The identified independent variables are presented in table \ref{table:method:experiment:independent-variables}.

\todo{Mention lack of SSE, HMC, FPGA etc.? Argue that it is out of scope.}

\begin{table}[H]
    \centering
    \caption{Independent Variables}
    \label{table:method:experiment:independent-variables}
    \begin{tabularx}{\linewidth}{l>{\RaggedRight}l>{\RaggedRight}X>{\RaggedRight\arraybackslash}X}
        \toprule
        \thead{Group} & \thead{Label} & \thead{Potential Values} & \thead{Comment} \\
        \midrule
        Compiler & Compilers & GCC, Clang, IBM XL, Intel C++ Compiler (ICC) & There are more, but these are the most commonly used compilers\todo{Cite properly?} \\
        Compiler & Flags & -O0, -O1, ..., -O5, -march, -mtune, -fomit-frame-pointer, -fwrapv, -fPIC, -fPIE, -mcpu, -Qunused-arguments & There are virtually infinite possible flags. These are the minimal ones used for optimization in this context and as used by previous work (\gls{supercop}, \gls{nist} publications)\todo{Research correct flags} \\
        Implementation & Libraries & \gls{sha3} / \gls{shake} / \gls{keccak}, \gls{aes-instruction-set}, \gls{cpacf} & \\
        Implementation & Optimizations & \gls{avx}, \gls{avx2}, \gls{avx512}, \gls{simd} (GPU), \gls{simd} on \gls{ibmz}, \gls{hsm} & These are the optimizations available in the selected implementations \\
        Environment & OS & OS Type and version, Scheduling & \\
        Environment & Hardware & CPU model, CPU Architecture, available cores, available RAM, RAM model and specification & \\
        \bottomrule
    \end{tabularx}
\end{table}

In table \ref{table:method:experiment:dependent-variables}, the dependant variables of interest are presented. These are the variables that relate to the goal as specified in section \ref{section:method:experiment-goal}. Note that, as \gls{ecdh} and \gls{dh} are not \glspl{kem}, but \gls{kex} algorithms, its "encapsulation" and "decapsulation" is the same stage. As such, only one measurement is necessary and will henceforth be referred to as the encapsulation phase.

\begin{table}[H]
    \centering
    \caption{Dependent Variables}
    \label{table:method:experiment:dependent-variables}
    \begin{tabularx}{\linewidth}{l c X}
        \toprule
        \thead{Group} & \thead{Stages} & \thead{Comments}\\
        \midrule
        Throughput & \makecell{Encapsulation,\\Decapsulation,\\Key-pair generation} & CPU cycles, instruction count, wall-clock time\todo{Update? Branches, cache misses, page faults?}\\
        Hardware Utilization & \makecell{Encapsulation,\\Decapsulation,\\Key-pair generation} & Memory usage (heap and stack allocation), logical core utilization\footnotemark\\
        \bottomrule
    \end{tabularx}
\end{table}
\addtocounter{footnote}{-1}
\addtocounter{footnote}{1}
\footnotetext{Note that some of the chosen environments may not expose physical core, rather \glspl{vcpu} or logical cores.}

\subsection{Instrumentation}

This section describes the tools and instrumentation used in the experiment.

\todo[inline]{Förtydliga att hot-paths enbart utförs på referens-implementationen på en plattform. Förtydliga att minneskontroll enbart utförs på de enkla beräkningar och att ingen minneskontroll när vi kör i bulk. Minneskontroll görs inte vid multicore-benchmark p.g.a. att vi allokerar minne o.s.v.}

\subsubsection{Toolset - Processor}
\todo[inline]{Describe that there will be two different "experiments", one for single iteration and one for multi-core}
\subsubsubsection{Selected Tools}
The instrumentation will rely on the Linux perf (perf\_event\_open) API. This API is part of the kernel and produces highly accurate measurements, regardless of the underlying hardware. The recorded values for CPU cycles, instructions etc. may be counted or sampled. The API functions externally, by monitoring the entire lifetime of a process, or internally by instrumenting a program with specifically measured blocks of code.\todo{Cite this properly?}

As such the Linux perf API was identified as the only applicable tool for profiling the CPU usage of an algorithm. To help aid the usage of the API, we developed a small library which we used to instrument the algorithms. The library is open source and published as part of this work\footnote{\href{https://github.com/profiling-pqc-kem-thesis/perf}{https://github.com/profiling-pqc-kem-thesis/perf}}.

Valgrind (Callgrind) was identified as usable for identifying hot paths in the code.

\noindent
\subsubsubsection{Rejected Tools}
Other sampling-based measurement alternatives that require the tested binary to be instrumented, such as prof, gprof and gperftools are unable to accurately measure IO-bound tasks, the retired instructions or time spent inside of the kernel\footnote{\href{https://ftp.gnu.org/old-gnu/Manuals/gprof-2.9.1/html\_mono/gprof.html}{https://ftp.gnu.org/old-gnu/Manuals/gprof-2.9.1/html\_mono/gprof.html}}

Another, non-sampling-based tool that monitor the entire lifetime of a process, Valgrind and its sub-tool Callgrind was found to not support running all of our samples on all targeted environments and was therefore deemed inappropriate for use as the main instrumentation tool. Callgrind can and will however be used for identifying the hot paths of the reference implementations as the implementation can be run in a working environment with its results general to wherever the implementation may be run.\todo{Revisit Valgrind comments}

Other libraries or APIs such as those available in \gls{supercop} and PAPI were identified via testing and a source code analysis to provide inaccurate and non-deterministic results and were therefore not applicable for this use case. In the case of \gls{supercop}, some measurements were identified to rely on extrapolation and estimation of data which led us to reject the tool.

\subsubsection{Toolset - Memory}
\todo[inline]{Describe that there will be two different "experiments", one for single iteration and one for multi-core}
\subsubsubsection{Selected Tools}
As previously mentioned, Valgrind is a non-sampling-based tool which enables it to provide accurate measurements. Although we found it was not applicable for us to use for processor profiling, it was identified as the tool of choice\todo{Revisit Valgrind comments} for memory analysis.

\noindent
\subsubsubsection{Rejected Tools}
\todo[inline]{vTune????}

\subsubsection{Measurements}

\todo[inline]{Describe that there will be two different "experiments", one for single iteration and one for multi-core}
\todo[inline]{Describe what each measurement is and what we aim to get out of them. Describe what system, user and wall-clock time is}

Using the aforementioned toolset, we will instrument the subjects and measure the following values.

\begin{itemize}
    \item Retired instruction count
    \item Total cycle count
    \item Page fault count
    \item Cache miss count \footnote{x86 only - not available on s390x}
    \item Heap allocation in bytes
    \item Stack usage in bytes
    \item Wall clock time in nanosecond resolution
\end{itemize}\todo{Describe why these are not the same as in Table 1.4}

The measurements will be taken at different points in the algorithms' lifecycles as explained in table \ref{table:method:instrumentation}. The same measurements are to be repeated for the three phases - key-generation, encryption (key encapsulation) and decryption (key decapsulation). By first analyzing the hot paths of the \gls{nist} submissions, we will also measure the parts of the algorithm that are invoked the most. As some variables in the environment have been identified, such as the various implementations for \gls{shake} and \gls{aes}, the use of these will be measured as well. In some cases, the authors have provided optimized implementations. In such cases, these optimized functions will be measured as well.

\begin{table}[H]
    \centering
    \caption{Measurement Locations}
    \label{table:method:instrumentation}
    \begin{tabularx}{\linewidth}{l>{\RaggedRight}X>{\RaggedRight\arraybackslash}X}
        \toprule
        \thead{Subject} & \thead{Location} & \thead{Comment}\\
        \midrule
        All & Right before and right after the algorithm is invoked & Provides the bounds of the timeline of the execution \\
        \gls{mceliece} & Right before and right after the matrix multiplication is invoked & \\
        \gls{mceliece}, \gls{ntru} & Right before and right after random bytes are requested & \\
        \gls{mceliece}, \gls{ntru} & Right before and right after the SHAKE implementation is called & \\
        \gls{mceliece} & int32\_sort & \\
        \gls{mceliece} & syndrome, transpose, vec256\_sq, vec256\_copy, ... &\\
        \gls{ntru} & Poly math - ntru\_poly\_S3\_inv, ntru\_poly\_rq\_to\_s3, ntru\_poly\_rq\_mul, square, ...& \\
        \gls{ntru} & crypto\_sort\_int32 & \\
        \bottomrule
    \end{tabularx}
\end{table}
\todo{Update with accurate information}

For the exact instrumentation used in each subject, please refer to the source code published alongside this work\footnote{\href{https://github.com/profiling-pqc-kem-thesis/subjects}{https://github.com/profiling-pqc-kem-thesis/subjects}}.

\subsection{Execution}

This section describes how the experiment will be executed, including the configurations used for testing.

\subsubsection{Hardware Configurations}

To fulfil the goal of the experiment, several hardware configurations will be tested. These configurations have been divided into three categories, namely Consumer Hardware, Server Hardware and Mainframe Hardware.

\begin{table}[H]
    \centering
    \caption{Consumer Hardware - x86}
    \label{table:method:consumer-hardware}
    \begin{tabularx}{\linewidth}{X c c}
        \toprule
        \thead{Label} & \thead{CPU} & \thead{RAM}\\
        \midrule
        Modern Workstation & Intel i9-9900k & 32GB 3600MHz DDR4\\
        Modern Laptop & Intel i7-8565u & 16 GB 2600MHz DDR4\\
        Old Mid-Range Laptop & Intel i5-3230m & 8GB 1600MHz DDR3\\
        Old Low-Range Laptop & Intel i3-3120m & 8GB 1600MHz DDR3\\
        \bottomrule
    \end{tabularx}
\end{table}
\todo{The above table requires more up-to-date information.}

\begin{table}[H]
    \centering
    \caption{Server Hardware - x86}
    \label{table:method:server-hardware}
    \begin{tabularx}{\linewidth}{X c c}
        \toprule
        \thead{Label} & \thead{CPU} & \thead{RAM}\\
        \midrule
        Cloud Provider 1 - Shared CPU\footnotemark & 1 vCPU & 1GB RAM \\
        Cloud Provider 1 - Dedicated CPU\footnotemark & 1 CPU & 1GB RAM \\
        Cloud Provider 2 - Shared CPU\footnotemark & 1 vCPU & 1GB RAM \\
        Cloud Provider 2 - Dedicated CPU\footnotemark & 1 CPU & 1GB RAM \\
        \bottomrule
    \end{tabularx}
\end{table}
\addtocounter{footnote}{-4}
\addtocounter{footnote}{1}
\footnotetext{A DigitalOcean "small droplet"}
\addtocounter{footnote}{1}
\footnotetext{A DigitalOcean "small droplet"}
\addtocounter{footnote}{1}
\footnotetext{A Google Cloud Platform "small droplet"}
\addtocounter{footnote}{1}
\footnotetext{A Google Cloud Platform "small droplet"}
\todo{The above table requires more up-to-date information.}

\begin{table}[H]
    \centering
    \caption{Mainframe Hardware}
    \label{table:method:mainframe-hardware}
    \begin{tabularx}{\linewidth}{X c c c}
        \toprule
        \thead{Label} & \thead{Model} & \thead{CPU} & \thead{RAM}\\
        \midrule
        Mainframe & IBM \gls{z15} & \gls{z15} (\gls{s390x}) & 4GB\\
        \bottomrule
    \end{tabularx}
\end{table}
\todo{The above table requires more up-to-date information.}

\subsubsection{Compiler Configurations}

The following compiler configurations will be assessed where applicable.

\begin{table}[H]
    \centering
    \caption{Compiler Configurations}
    \label{table:method:compilers}
    \begin{tabularx}{\linewidth}{l c c X}
        \toprule
        \thead{Label} & \thead{Name} & \thead{Version} & \thead{Flags}\\
        \midrule
        GCC Minimal & GCC & 8 & Minimal required\\
        Clang Minimal & Clang & 8 & Minimal required\\
        IBM XL Minimal & IBM XL & 8 & Minimal required\\
        GCC Optimized & GCC & 8 & -march=native\\
        Clang Optimized & Clang & 8 & -march=native\\
        IBM XL Optimized & IBM XL & 8 & -march=native\\
        \bottomrule
    \end{tabularx}
\end{table}

\todo[inline]{-march=native räcker väl inte för att optimera mycket? Vilka kommer ni egentligen att använda?}

In order to see the exact flags used to compile each sample, please refer to the source code for the experiment\footnote{\href{https://github.com/profiling-pqc-kem-thesis}{https://github.com/profiling-pqc-kem-thesis}}.

\subsubsection{Implementation Configurations}

The following implementation details will be changed in order to evaluate potential performance differences.

\begin{table}[H]
    \centering
    \caption{Implementation configurations}
    \label{table:method:implementation-configurations}
    \begin{tabularx}{\linewidth}{l>{\RaggedRight}X>{\RaggedRight\arraybackslash}X}
        \toprule
        \thead{Label} & \thead{Description}\\
        \midrule
        Reference Implementation & The reference implementation as published in the \gls{nist} submissions, or the available \gls{openssl} implementation. \\
        Optimized Implementation(s) & Use the authors' provided optimizations.\\
        Optimized \gls{shake} implementation & Use a \gls{keccak} implementation optimized for the native platform (plain, \gls{avx2}, \gls{avx512}).\\
        Optimized \gls{aes} implementation & Exchange the included \gls{aes} implementation with \gls{openssl} which is optimized for the native platform (plain, \gls{aes}, \gls{aes-ni}).\\
        Fully optimized implementation & Optimized implementation, optimized \gls{shake} implementation, optimized \gls{aes} implementation.\\
        \bottomrule
    \end{tabularx}
\end{table}
\todo{The above table requires more up-to-date information.}

\subsubsection{Configuration Summary}

The configurations presented in table \ref{table:method:configuration-summary} summarizes the configurations that are to be tested.

\todo[inline]{Describe that there will be two different "experiments", one for single iteration and one for multi-core}

\begin{table}[H]
    \centering
    \caption{Configuration Summary}
    \label{table:method:configuration-summary}
    \begin{tabularx}{\linewidth}{l X X}
        \toprule
        \thead{Hardware} & \thead{Compiler Configurations} & \thead{Implementation Configurations}\\
        \midrule
        Modern Workstation & GCC Minimal, GCC Optimized, Clang Minimal, Clang Optimized & All (\gls{avx2}, \gls{aes-instruction-set})\\
        Modern Laptop & GCC Minimal, GCC Optimized, Clang Minimal, Clang Optimized & All (\gls{avx2}, \gls{aes-instruction-set}) \\
        Old Mid-Range Laptop & GCC Minimal, GCC Optimized, Clang Minimal, Clang Optimized & All (\gls{avx}, \gls{aes-instruction-set})\\
        Old Low-Range Laptop & GCC Minimal, GCC Optimized, Clang Minimal, Clang Optimized & All (\gls{avx})\\
        Cloud Provider 1 - Shared CPU & GCC Minimal, GCC Optimized, Clang Minimal, Clang Optimized & All (\gls{avx512}, \gls{aes-ni})\\
        Cloud Provider 1 - Dedicated CPU & GCC Minimal, GCC Optimized, Clang Minimal, Clang Optimized & All (\gls{avx512}, \gls{aes-ni})\\
        Cloud Provider 2 - Shared CPU & GCC Minimal, GCC Optimized, Clang Minimal, Clang Optimized & All (\gls{avx512}, \gls{aes-ni})\\
        Cloud Provider 2 - Dedicated CPU & GCC Minimal, GCC Optimized, Clang Minimal, Clang Optimized & All (\gls{avx512}, \gls{aes-ni})\\
        Mainframe & GCC Minimal, GCC Optimized, IBM XL, IBM XL Optimized & All (\gls{cpacf})\\
        \bottomrule
    \end{tabularx}
\end{table}