\abstract
% Introduction, som i conclusion fast ännu mer sammanfattat
% Dagsläget, problemet, lösningen (pqc)
\noindent\textbf{Background.} People use the Internet for communication, work, online banking and more. Public-key cryptography enables this use to be secure by providing confidentiality and trust online. Though these algorithms may be secure from attacks from classical computers, future quantum computers may break them using Shor's algorithm. Therefore \gls{post-quantum} algorithms are being developed to mitigate this issue. The \acrlong{nist} has started a standardization process for these algorithms.\newline
% Sammanfatta våra research questions. "We analyze prestanda o.s.v., ej nämna rqs"
\textbf{Objectives.} In this work, we analyze what specialized features applicable for \gls{post-quantum} algorithms are available in the mainframe architecture \gls{ibmz}. Furthermore, we study the performance of these algorithms on various hardware in order to understand what techniques may increase their performance.\newline
% Litterature study, experimental study - lite sammanfattning från metoden? Vi har ett stycke om våra metod
\textbf{Methods.} We apply a literature study to identify the performance characteristics of \gls{post-quantum} algorithms as well as what features of \gls{ibmz} may accommodate and accelerate these. We further apply an experimental study to analyze the practical performance of the two prominent finalists \gls{ntru} and \gls{mceliece} on consumer, cloud and mainframe hardware.\newline
% Kortfattat från rq1-3 i conclusions
\textbf{Results.} \gls{ibmz} was found to be able to accelerate several key symmetric primitives such as \gls{sha3} and \gls{aes} via \gls{cpacf}. Though the available \acrlong{hsm}s did not support any of the studied algorithms, it was found to be able to accelerate them via a \gls{fpga}. Based on our experimental study, we found that computers with support for the Advanced Vector Extensions (\gls{avx}) were able to significantly accelerate the execution of \gls{post-quantum} algorithms. Lastly, we identified that vector extensions, \glspl{asic} and \glspl{fpga} are key techniques for accelerating these algorithms.\newline
% Outlook-stycket?
\textbf{Conclusions.} When considering the readiness of hardware for the transition to \gls{post-quantum} algorithms, we find that the proposed algorithms do not perform nearly as well as classical algorithms. Though the algorithms are likely to improve until the transition occurs, improved hardware support via faster vector instructions, increased cache sizes and the addition of polynomial instructions may significantly help reduce the impact of the \gls{post-quantum} transition.

\vspace{1cm}
\noindent
\textbf{Keywords:} Public-Key Cryptography, Benchmark, \gls{x86}, \gls{ibmz}, \gls{z15}

\todo[inline]{
inconsistencies in how KECCAK, SHAKE and SHA-3 is used?
}

\todo[inline]{
Describe ref-optimized etc. in the method?

Typeset inline-code and compiler flags?
}

\todo[inline]{
Kolla igenom alla glossaries - om de används (check\_glossaries.py)
}

\todo[inline]{
Fix noindent in applicable places.

Fix table and graph locations - from the top down.

Clarify how ECDHE and DHE are implemented - one keypair end exchange per "exchange" phase.

Chapters in appendix? Order it?
}

\cleardoublepage