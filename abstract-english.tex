\abstract
Most readers will turn first to the abstract of your thesis. Use it as an opportunity to spur the reader's interest. The abstract should highlight the main points from your work, especially the thesis' problems statement, methods, findings and conclusions. However, the abstract does not need to cover every aspect of your work. The main objective is to give the reader a good idea of what the thesis is about.

The abstract should be completed towards the end, when you are able to overview your project as a whole. It is nevertheless a good idea to work on a draft continuously. Writing a good abstract can be difficult, since it should only include the most important points of your work. But this is also why working on your abstract can be so useful -- it forces you to identify the key elements of your degree project.

Structured abstracts have several advantages for authors and readers. They help readers to quickly find information in an abstract and also guide authors in summarizing the content of their manuscripts precisely. Below you find the main components of a structured abstract.

\noindent
\textbf{Background.} ... \newline
\textbf{Objectives.} ... \newline
\textbf{Methods.} ... \newline
\textbf{Results.} ... \newline
\textbf{Conclusions.} ...

\vspace{1cm}
% You can list up to 5 keywords, at most 2 appearing in the title;
% starts 1 line below the abstract.
\noindent
\textbf{Keywords:} Public-Key Cryptography, Benchmark, x86, IBM Z, z15

\todo[inline]{
inconsistencies in how KECCAK, SHAKE and SHA-3 is used?
}

\todo[inline]{
Describe ref-optimized etc. in the method?

Typeset inline-code and compiler flags?
}

\todo[inline]{
Make all tables small?
}

\todo[inline]{
Kolla igenom alla glossaries - om de används (check\_glossaries.py)
}

\todo[inline]{
Fix noindent in applicable places.
}

\todo[inline]{
Fix table and graph locations - from the top down.
}

\todo[inline]{
Clarify how ECDHE and DHE are implemented - one keypair end exchange per "exchange" phase.
}

\todo[inline]{
Change all " gls{..}" to "~gls{..}
}

\todo[inline]{
Chapters in appendix? Order it?
}

\todo[inline]{
Conclusion validity
}

\cleardoublepage